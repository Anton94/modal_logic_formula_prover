% This text is proprietary.
% It's a part of presentation made by myself.
% It may not used commercial.
% The noncommercial use such as private and study is free
% Dec 2007
% Author: Sascha Frank 
% University Freiburg 
% www.informatik.uni-freiburg.de/~frank/
%
% 
\documentclass{beamer}
\usepackage[utf8]{inputenc}
\usepackage[T2A,T1]{fontenc}
\usepackage[english, bulgarian]{babel}
\setbeamertemplate{navigation symbols}{}

\addtobeamertemplate{navigation symbols}{}{%
    \usebeamerfont{footline}%
    \usebeamercolor[fg]{footline}%
    \hspace{1em}%
    \insertframenumber/\inserttotalframenumber
}

\usepackage{mathtools}
\usepackage{amsmath}
\usepackage{amssymb}
\usepackage{amsthm}
\usepackage{systeme}
\usepackage{newtxtext}
\usepackage{newtxmath}
\usepackage{listings}
\usepackage{xcolor}
\usepackage{array}
\usepackage{makecell}
\usepackage{booktabs}% http://ctan.org/pkg/booktabs
\usepackage{float}
\usepackage{hyperref}
\usepackage{tikz}
\usepackage{multicol}
\usepackage{graphicx}

\usetikzlibrary{arrows.meta}
\graphicspath{ {./image/} }

\restylefloat{table}
\newcommand{\tabitem}{~~\llap{\textbullet}~~}

\newtheorem{defn}{Дефиниция}[section]
\newtheorem{axiom}{Аксиома}[section]
\newtheorem{prop}{Свойство}[section]
\newtheorem{lema}{Лема}[section]

\newcommand{\curvedE}{\mathscr{e}}
\newcommand{\vVar}{\mathscr{v}_{var}}
\newcommand{\Var}{\mathbb{V}ar}
\newcommand{\vE}{\mathscr{v}}
\newcommand{\vBool}{\xi}
\newcommand{\Ts}{\mathcal{T}_s}
\newcommand{\signT}{\mathbb{T}}
\newcommand{\signF}{\mathbb{F}}
\newcommand{\BranchConjunction}{%
\begin{align*}
			\bigwedge_{i=1}^{I} \; C(a_i, b_i) \:\: & \wedge \:\: 
			\bigwedge_{k=1}^{K} \; \neg C(e_k, f_k) \:\: \wedge \:\: \\
			\bigwedge_{j=1}^{J} \; d_j \neq 0 \:\: & \wedge \:\:
			\bigwedge_{l=1}^{L} \; g_l = 0 \:\:
\end{align*}%
}
\newcommand{\BranchConjunctionWithMeasure}{%
\begin{align*}
			\bigwedge_{i=1}^{I} \; C(a_i, b_i) \:\: & \wedge \:\: 
			\bigwedge_{k=1}^{K} \; \neg C(e_k, f_k) \:\: \wedge \:\: \\
			\bigwedge_{j=1}^{J} \; d_j \neq 0 \:\: & \wedge \:\:
			\bigwedge_{l=1}^{L} \; g_l = 0 \:\: \wedge \:\:\\
			\bigwedge_{p=1}^{P} \; m_p \le_\mu n_p \:\: &\wedge \:\:
			\bigwedge_{q=1}^{Q} \; u_q <_\mu v_q \:\:
\end{align*}%
}

\lstset { %
    literate={~} {$\sim$}{1},
    language=C++,
    backgroundcolor=\color{black!5}, % set backgroundcolor
    basicstyle=\footnotesize,% basic font setting
}



\usetheme{Montpellier}

\beamersetuncovermixins{\opaqueness<1>{25}}{\opaqueness<2->{15}}
\begin{document}
\title{Уеб система за изпълнимост в контактната логика с мярка}  
\author{\textsc{Факултет по математика и информатика} \\
	\textsc{Катедра по математическа логика и приложенията ѝ} \\ [3mm]
	\textsc{Стоев Мартин} \\ [3mm]
	\textsc{Магистрерска програма Логика и Алгоритми, Информатика, Факултетен Номер: 25790} \\ [4mm]
	\small\textsc{Научен ръководител: проф. Тинко Тинчев}}
\date{\today} 
\begin{frame}
\titlepage
\end{frame}

%\begin{frame}\frametitle{Съдържание}\tableofcontents\end{frame} 


\section{Въведение} 
\begin{frame}\frametitle{Въведение}
\begin{itemize}
	\item Каква е целта на тази дипломна работа ?
	\item Теоретична част
	\item Практическа част
\end{itemize}
\end{frame}

\section{Съдържание} 
\begin{frame}\frametitle{Съдържание}
\begin{itemize}
	\item Табло метод
	\item Kонтактна логика
	\item Изпълнимост в контактна логика
	\item Kонтактна логика с мярка
	\item Изпълнимост в контактна логика с мярка
\end{itemize}
\end{frame}

\section{Табло метод}
\begin{frame}\frametitle{Табло метод}
\begin{itemize}
	\item Табло метод като процедура за опровергаване на формули
	\item Табло метод като процедура за построяване на модели
	\item Табло метод за съждителната логика
\end{itemize}
\end{frame}

\subsection{Табло метод за съждителната логика}
\begin{frame}\frametitle{Табло метод за съждителната логика}
Маркиране на валидността на формула $\varphi$
\begin{itemize}
	\item $\signT \varphi$ - маркиране на формулата $\varphi$ за вярна
	\item $\signF \varphi$ - маркиране на формулата $\varphi$ за невярна
\end{itemize}
\vspace{20px}
Стъпки на табло метода
\begin{itemize}
	\item Разширяване на дърво, което строи табло методът
	\item Намиране на противоречия
\end{itemize}
\end{frame}

\begin{frame}\frametitle{Правила}
Отрицание
\vspace{20px}
		\noindent\begin{minipage}{.5\linewidth}
		\begin{align*}
			\begin{array}{rl}
				& \signT(\neg \varphi), X \\
				      \cline{2-2}
				      & \signF(\varphi), X
			\end{array}
		\end{align*}
		\end{minipage}%
		\begin{minipage}{.5\linewidth}
		\begin{align*}
			\begin{array}{rl}
				& \signF(\neg \varphi), X \\
			      \cline{2-2}
			      & \signT(\varphi), X
			\end{array}
		\end{align*}
		\end{minipage}
Конюнкция
		\noindent\begin{minipage}{.5\linewidth}
		\begin{align*}
			\begin{array}{rl}
				& \signT(\varphi \: \wedge \: \psi), X \\
			      \cline{2-2}
			      & \signT\varphi, \signT\psi, X
			\end{array}
		\end{align*}
		\end{minipage}%
		\begin{minipage}{.5\linewidth}
		\begin{align*}
			\begin{array}{rl}
				& \:\:\: \signF(\varphi \: \wedge \: \psi), X \\
			      \cline{2-2}
			      & \signF\varphi, X \:\:\:\:\:\:\:\: \signF\psi, X
			\end{array}
		\end{align*}
		\end{minipage}
\end{frame}

\begin{frame}\frametitle{Правила}
Дизюнкция
\vspace{20px}
		\noindent\begin{minipage}{.5\linewidth}
		\begin{align*}
			\begin{array}{rl}
				& \:\:\: \signT(\varphi \: \vee \: \psi), X \\
			      \cline{2-2}
			      & \signT\varphi, X \:\:\:\:\:\:\:\: \signT\psi, X
			\end{array}
		\end{align*}
		\end{minipage}%
		\begin{minipage}{.5\linewidth}
		\begin{align*}
			\begin{array}{rl}
				& \signF(\varphi \: \vee \: \psi), X \\
			      \cline{2-2}
			      & \signF\varphi, \signF\psi, X
			\end{array}
		\end{align*}
		\end{minipage}
Импликация
		\noindent\begin{minipage}{.5\linewidth}
		\begin{align*}
			\begin{array}{rl}
				& \:\:\: \signT(\varphi \: \rightarrow \: \psi), X \\
			      \cline{2-2}
			      & \signF\varphi, X \:\:\:\:\:\:\:\: \signT\psi, X
			\end{array}
		\end{align*}
		\end{minipage}%
		\begin{minipage}{.5\linewidth}
		\begin{align*}
			\begin{array}{rl}
				& \signF(\varphi \: \rightarrow \: \psi), X \\
			      \cline{2-2}
			      & \signT\varphi, \signF\psi, X
			\end{array}
		\end{align*}
		\end{minipage}
\end{frame}

\begin{frame}\frametitle{Правила}
Еквивалентност
		\noindent\begin{minipage}{.5\linewidth}
		\begin{align*}
			\begin{array}{rl}
				& \:\:\:\:\:\:\:\: \signT(\varphi \: \leftrightarrow \: \psi), X \\
			      \cline{2-2}
			      & \signT\varphi, \signT\psi, X \:\:\:\:\:\:\:\: \signF\varphi, \signF\psi, X
			\end{array}
		\end{align*}
		\end{minipage}%
		\begin{minipage}{.5\linewidth}
		\begin{align*}
			\begin{array}{rl}
				& \:\:\:\:\:\:\:\: \signF(\varphi \: \leftrightarrow \: \psi), X \\
			      \cline{2-2}
			      & \signT\varphi, \signF\psi, X \:\:\:\:\:\:\:\: \signF\varphi, \signT\psi, X
			\end{array}
		\end{align*}
		\end{minipage}
\end{frame}

\begin{frame}\frametitle{Дефиниции}

\begin{defn}[Затворен клон]
Когато в него има едновременно една и съща формула маркирана за вярна и за невярна.
\end{defn}

\begin{defn}[Атомарен клон]
Когато клона не може да се разширява повече.
\end{defn}

\begin{defn}[Завършено табло]
Когато всеки клон в таблото е или затворен или атомарен.
\end{defn}
\end{frame}

\begin{frame}\frametitle{Общовалидна формула}
	Проверяваме дали дадена формула $\varphi$ е общовалидна с следните стъпки:
\begin{enumerate}
	\item Маркираме $\varphi$ за невярна, т.е.  $\signF\varphi$.
	\item Ползваме $\signF\varphi$ за начална формула на таблото.
	\item Разширяваме докато таблото не е завършено.
	\item Ако всички клонове на таблото са затворени, то формулата $\varphi$ е общовалидна.
\end{enumerate}
\end{frame}

\begin{frame}\frametitle{Пример}
	\hspace{50px} 1. $\signF (X \rightarrow ((X \wedge \neg Y) \vee \neg X)$ \\
	\hspace{50px} 2. $\signT X, \signF ((X \wedge \neg Y) \vee \neg X)$ \\
	\hspace{50px} 3. $\signT X, \signF (X \wedge \neg Y), \signF \neg X$ \\
	\hspace{50px} 4. $\signT X, \signF (X \wedge \neg Y), \signT X$ \\
	\begin{tikzpicture}[scale = 0.8]
		\draw (-7,0) (5,1);
        \draw[thick] (-3.1,0) -- (-2.2,1) -- (-1.3,0);
    \end{tikzpicture} \\
	\hspace{70px} 5. $\signF X$ \hspace{5px} 6. $\signF \neg Y$\\
	\hspace{108px} 7. $\signT Y$
\end{frame}

\section{Kонтактна логика}
\begin{frame}\frametitle{Kонтактна логика}
\begin{enumerate}
	\item Синтаксис 
	\item Семантика
	\item Свойства
	\item Изпълнимост на формула
\end{enumerate}
\end{frame}

\subsection{Синтаксис}
\begin{frame}\frametitle{Синтаксис}
	\begin{itemize}
		\item W - цял свят
		\item $\emptyset$ - празен регион
		\item $\mathbb{V}ar$ - изброимо множество от променливи
		\item Булеви константи за W и $\emptyset$, 1 и 0 съответно
	\end{itemize}
\end{frame}

\begin{frame}\frametitle{Булеви операции}
	\begin{itemize}
		\item $\sqcap$ за булево сечение
		\item $\sqcup$ за булево обединение
		\item * за допълнение
	\end{itemize}
\end{frame}

\begin{frame}\frametitle{Дефиниция за терм}
Терм се дефинира индуктивно:
	\begin{itemize}
		\item Булевите константи са термове
		\item $p \in \mathbb{V}ar$ е терм
		\item Ако x е терм, то x* е също така терм
		\item Ако x и y са два терма, то $x \; \sigma \; y $ е също така терм,\\
				където $\sigma \: \in \: \{\sqcap, \sqcup\}$
	\end{itemize}
\end{frame}

\begin{frame}\frametitle{Константи, операции и атомарни формули}
	Съждителни константи: $\top$ and $\bot$ \\
	\vspace{10px}
	Съждителни операции: $\neg, \vee, \wedge, \rightarrow, \leftrightarrow$
	\vspace{10px}

	Нека a и b са два терма. То тогава 
	\begin{itemize}
		\item C(a, b)
		\item $a \le b$
	\end{itemize}
	са атомарни формули.
\end{frame}

\begin{frame}\frametitle{Дефиниция за формула}
Формула се дефинира индуктивно:
		\begin{itemize}
			\item Всяка съждителна константа е формула
			\item Всяка атомарна формула е формула
			\item Ако $\varphi$ е формула, то $\neg{\varphi}$ е също така формула
			\item Ако $\varphi$ и $\psi$ са две формули, то $\varphi \: \sigma \: \psi $ е също така формула,\\
				където $\sigma \in \{\vee, \wedge, \rightarrow, \leftrightarrow\}$
		\end{itemize}
\end{frame}

\begin{frame}\frametitle{Съкращения}
		\begin{itemize}
		\item $a = b\:\textit{означава}\:\: (a \le b) \land (b \le a)$
		\item $a \neq b\:\textit{означава}\:\: \neg (a = b) $
		\item $a \nleq b\:\textit{означава}\:\: \neg (a \le b)$
		\end{itemize}
\end{frame}

\subsection{Семантика}
\begin{frame}\frametitle{Семантика}
	Релационна система се дефинира като $(W, R)$, където $W \neq \emptyset$, R е рефлексивна и симетрична релация в W.
	\newline
	\newline
	Булева оценка на променливите означаваме с  $\mathscr{v}: \mathbb{V}ar \rightarrow  \mathscr{P}(W)$ и разширяваме я до множеството на всички термове:
		\begin{itemize}
			\item $\mathscr{v}(0) = \emptyset$
			\item $\mathscr{v}(1) = W$
			\item $\mathscr{v}(a \sqcap b) = \mathscr{v}(a) \cap \mathscr{v}(b)$
			\item $\mathscr{v}(a \sqcup b) = \mathscr{v}(a) \cup \mathscr{v}(b)$
			\item $\mathscr{v}(a*) = W \backslash  \mathscr{v}(a)$
		\end{itemize}
	
\end{frame}

\subsection{Семантика}
\begin{frame}\frametitle{Модел}
	Наредената тройка $\mathcal{M} = (W, R, \mathscr{v})$ наричаме модел. Дефинираме вярност на дадена формула в $\mathcal{M}$, като:
		\begin{itemize}
			\item $\mathcal{M} \not\models \bot$ 
			\item $\mathcal{M} \models \top$
			\item $\mathcal{M} \models aCb \text{ когато } (\exists x \in \mathscr{v}(a)) (\exists y \in \mathscr{v}(b)) (xRy)$
			\item $\mathcal{M} \models a \leq b \text{ когато } \mathscr{v}(a) \subseteq \mathscr{v}(b)$
			\item $\mathcal{M} \models \neg \varphi \text{ когато } \mathcal{M} \not\models \varphi$
			\item $\mathcal{M} \models a \; \vee \; b \text{ когато } \mathcal{M} \models a \: \textit{ или } \: \mathcal{M} \models b$
			\item $\mathcal{M} \models a \; \wedge \; b \text{ когато } \mathcal{M} \models a \: \textit{ и } \: \mathcal{M} \models b$
		\end{itemize}
	
\end{frame}

\subsection{Свойства}
\begin{frame}\frametitle{Свойства}

\begin{prop}[Рефлексивност на контакта]
Нека b е терм, тогава:
	\begin{equation*}
		\mathcal{M} \models b \neq 0  \iff \mathcal{M} \models bCb
	\end{equation*}
\end{prop}

\begin{prop}[Симетрия на контакта]
Нека a и b са два терма, тогава:
	\begin{equation*}
		\mathcal{M} \models aCb \iff \mathcal{M} \models bCa
	\end{equation*}
\end{prop}

\begin{prop}[Еквивалентност на термове]
Нека a и b са два терма и $\mathscr{v}$ е оценка , тогава:
	\begin{equation*}
		\mathcal{M} \models a = b \iff \mathscr{v}(a) = \mathscr{v}(b)
	\end{equation*}
\end{prop}

\end{frame}
\begin{frame}\frametitle{Свойства}
\begin{prop}[Нулева формула]

Нека a и b са два терма, тогава:
	\begin{equation*}
		\mathcal{M} \models a \le b \iff \mathcal{M} \models a \sqcap b* = 0
	\end{equation*}
\end{prop}

\begin{prop}[Ненулева формула]
Нека a и b са два терма, тогава:
	\begin{equation*}
		\mathcal{M} \models \neg(a \le b) \iff \mathcal{M} \models a \sqcap b* \neq 0
	\end{equation*}
\end{prop}

\end{frame}


\subsection{Изпълнимост в контактна логика}
\begin{frame}\frametitle{Изпълнимост в контактна логика}
За дадена формула $\varphi$ трябва да построим модел $\mathcal{M} \models \varphi$.
\newline
\newline
Използваме табло метода за да опростим формулата и търсим модел в клоните на табло метода.
\newline
\newline
		Нека $\varphi$ е формула и $\mathcal{T}$ е таблото от $\varphi$, то клонът на таблото има следните маркирани атомарни формули:
\begin{itemize}
	\item $\signT C(a, b)$
	\item $\signF C(a, b)$ 
	\item $\signT t = 0$
	\item $\signF t = 0$
\end{itemize}
\end{frame}

\begin{frame}\frametitle{Табло клон}
		За улеснение можем да махнем $\signT \textit{ и } \signF$ и получаваме.
\begin{align*}
			\bigwedge_{i=1}^{I} \; C(a_i, b_i) \:\: & \wedge \:\: 
			\bigwedge_{k=1}^{K} \; \neg C(e_k, f_k) \:\: \wedge \:\: \\
			\bigwedge_{j=1}^{J} \; d_j \neq 0 \:\: & \wedge \:\:
			\bigwedge_{l=1}^{L} \; g_l = 0 \:\:
\end{align*}%
\end{frame}

\begin{frame}
\begin{defn}[Множеството на всички променливи]
	С $\Var_B$ ще ознчаваме множеството от всички променливи използвани в табло клона B.
\end{defn}

\begin{defn}[Оценка на променливи]
	С p ще ознчаваме функцията, която за всяка променлива от $\Var_B$ дава истина или лъжа.
		\begin{align*}
			p : \Var_B \rightarrow \{ \textbf{лъжа}, \textbf{истина}\}
		\end{align*}
\end{defn}
\end{frame}

\begin{frame}
\begin{defn}[Булева оценка]
		Нека p е оценка на променливи и $\Ts$ е множеството от всички термове, тогава 
		функцията $\vBool_p : \Ts \rightarrow \{ \textbf{лъжа}, \textbf{истина}\}$ ще наричаме булева оценка, която се дефинира по следния начин:
		\begin{itemize}
			\item $\vBool_p(0) = \textbf{лъжа}$
			\item $\vBool_p(1) = \textbf{истина}$
			\item $\vBool_p(t) =p(t), \textit{ където } t \in \Var_B$
			\item $\vBool_p(a \sqcap b) = \vBool_p(a) \textit{ и } \vBool_p(b)$
			\item $\vBool_p(a \sqcup b) = \vBool_p(a) \textit{ или } \vBool_p(b)$
			\item $\vBool_p(a*) = \textit{не } \vBool_p(a)$
		\end{itemize}
\end{defn}
\end{frame}

\begin{frame}\frametitle{Стъпки за построяване на модел}
	Нека групираме атомарните формули на такива, за които е необходимо съществуването на модална точка, и на такива, за които не е.
\newline
	Следните атомарни формули се нуждаят от същестуването на поне една модална точка:
		\begin{itemize}
			\item $C(a_i, b_i), \textit{ for } i < I$
			\item $d_j \neq 0, \textit{ for } j < J$
		\end{itemize}
\end{frame}

\begin{frame}\frametitle{Построяване на модални точки за контактите}
	За всеки контакт $C(a, b) \in B$ ще построим по две модални точки p и q, такива, че:
		\begin{itemize}
			\item $\vBool_{p}(a) = \textbf{истина}$
			\item $\vBool_{q}(b) = \textbf{истина}$
		\end{itemize}
\end{frame}

\begin{frame}\frametitle{Построяване на модални точки за контактите}
	Ново генерираните модални точки трябва да удовлетворяват не-контактите, т.е. следното условие трябва да е изпълнено:
		\begin{align*}
			 \neg C(e, f) \in B: (\vBool_{p}(e) = \textbf{лъжа} & \textit{ или } \vBool_{q}(f) = \textbf{лъжа}) \; \textit{ и } \\
				(\vBool_{p}(f) = \textbf{лъжа} & \textit{ или } \vBool_{q}(e) = \textbf{лъжа}) \; \textit{ и } \\
				(\vBool_{p}(e) = \textbf{лъжа} & \textit{ или } \vBool_{p}(f) = \textbf{лъжа}) \; \textit{ и } \\
				(\vBool_{q}(e) = \textbf{лъжа} & \textit{ или } \vBool_{q}(f) = \textbf{лъжа})
		\end{align*}

	Също така за равно на нула термовете, следното условие трябва да е изпълнено:
	\begin{align*}
		t = 0 \in B: \vBool_{p}(t) = \textit{лъжа} \text{ и }  \vBool_{q}(t) = \textit{лъжа}
	\end{align*}
\end{frame}

\begin{frame}\frametitle{Построяване на модални точки за контактите}
	След успешно генерираните модални точки за двата терма, разширяваме R със следните релации:
	\begin{itemize}
		\item pRp - рефлексивност на модалната точка p
		\item qRq - рефлексивност на модалната точка q
		\item pRq - симетричност между p и q
		\item qRp - симетричност между q и p
	\end{itemize}
\end{frame}

\begin{frame}\frametitle{Построяване на модални точки за контактите}
	\begin{center}
	\begin{tikzpicture}[scale = 1]
		\filldraw[color=black!60, fill=red!0, thick](-3,0) circle (0.5);
		\node at (-3,0) {p};
		\node at (-3.5,-0.7) {a};
		\draw[thick, ->] (-3.5,0) arc [
      	  		start angle=10,
		       end angle=335,
        		x radius=0.5cm,
        		y radius =0.2cm
    			] ;
		\draw[thick, <->](-2.5,0) -- (2.5,0);
		\filldraw[color=black!60, fill=red!0, thick](3,0) circle (0.5);
		\node at (3,0) {q};
		\node at (3.5,-0.7) {b};
		\draw[thick, ->] (3.5,0) arc [
      	  		start angle=180,
		      end angle=525,
        		x radius=0.5cm,
        		y radius =0.2cm
    			] ;
		\node[text width=10cm, anchor=west, right] at (-4.5,-2)
    		{Генерираните модални точки и техните релации за контакта C(a, b)};
      \end{tikzpicture}
	\end{center}
\end{frame}

\begin{frame}\frametitle{Построяване на модална точка за неравен на нула терм}
	За всеки неравен на нула терм $a \neq 0 \in B$ ще построим една модална точка, такава че $\vBool_{p}(a) = \textbf{истина}$. \\
	Аналогично, следните условия трябва да са изпълнени:
	\begin{align*}
		t = 0 \in B: \vBool_{p}(t) = \textit{лъжа}
	\end{align*}
	\begin{align*}
		\neg C(e, f) \in B: (\vBool_p(e) = \textbf{лъжа} \textit{ и } \vBool_p(f) = \textbf{лъжа})
	\end{align*}
	След успешно генериране на модалната точка, отново разширяваме R със следната:
	\begin{itemize}
		\item pRp - рефлексивност на модалната точка p
	\end{itemize}
\end{frame}

\begin{frame}\frametitle{Построяване на модална точка за не-равен на нула терм}
	\begin{center}
	\begin{tikzpicture}[scale = 1]
		\filldraw[color=black!60, fill=red!0, thick](-0,0) circle (0.5);
		\draw[thick, ->] (-0.5,0) arc [
      	  		start angle=10,
		       end angle=345,
        		x radius=0.5cm,
        		y radius =0.2cm
    			] ;
		\node at (-0,0) {p};
		\node at (-0.5,-0.7) {a};
		\node[text width=10cm, anchor=west, right] at (-4.5,-2)
    		{Генерираната модална точка и нейната релация за $a \neq 0$};
      \end{tikzpicture}
	\end{center} 
\end{frame}

\begin{frame}\frametitle{Построяване на модел}
	\begin{defn}
		Нека $\Ts$ е множеството от всички термове и нека (W, R) е релационата система, създадена с построяване на модални точки за контактите и неравно на нула термове.
		Тогава модалната оценка $\vE : \Ts \rightarrow \mathscr{P}(W)$ се дефинира рекурсивно, като:
		\begin{itemize}
			\item $\vE(0) = \emptyset$
			\item $\vE(1) = W$
			\item $\vE(t) = \{ p \; | \; p \in W \:\:\textit{и}\:\: p(t) = \textit{истина} \}$
			\item $\vE(a \sqcap b) = \vE(a) \cap \vE(b)$
			\item $\vE(a \sqcup b) = \vE(a) \cup \vE(b)$
			\item $\vE(a*) = W \setminus \vE(a)$
		\end{itemize}
	\end{defn}
\end{frame}

\begin{frame}\frametitle{Построяване на модел}
	\begin{lema}
		Нека a е терм и нека p е оценка на променливи, от дефиницията на $\vBool$ и $\vE$, следва, че:
		\begin{align*}
			\vBool_{p}(a) = \textit{ истина } \leftrightarrow p \in \vE(a)
		\end{align*}
	\end{lema}
\end{frame}

\section{Kонтактна логика с мярка}
\begin{frame}\frametitle{Kонтактна логика с мярка}
Kонтактна логика с мярка е самата контактна логика с добавена количествена мярка.
\newline
\newline
Мярката е функция, която на даден регион съпоставя положително реално число.
\begin{equation*}
	\mu : \mathscr{P}(W) \setminus \{\emptyset\} \longrightarrow \mathbb{R}^+
\end{equation*}
\end{frame}


\begin{frame}\frametitle{Семантика на контактната логика с мярка}
	Моделът на контактната логика се разширява със следната дефиниция за вярност:
	\begin{itemize}
		\item $\mathcal{M} \models a \leq_\mu b \text{, когато } \mu(\mathscr{v}(a)) \le \mu(\mathscr{v}(b))$
	\end{itemize}
\end{frame}

\begin{frame}\frametitle{Система линейни неравенства}
Формула с атомарни формули с мярка създава система линейни неравенства.
\newline
Системата линейни неравенства има следната структура:
		\[
			\syslineskipcoeff{1.50}
			\systeme*{\sum_{i^1} X_{i^1} <= \sum_{j^1} X_{j^1},
				\:\:.\:\:.\:\:.,
				\sum_{i^n} X_{i^n} <= \sum_{j^n} X_{j^n}, 
				\sum_{k^1} X_{k^1} > \sum_{l^1} X_{l^1},
				\:\:.\:\:.\:\:.,
				\sum_{k^m} X_{k^m} > \sum_{l^m} X_{l^m}
				}
		\]
\end{frame}

\begin{frame}\frametitle{Построяване на система линейни неравенства}
Нека $M = (W, R, \vE)$ е модел. Системата се построява с оценяване на термовете в $\le_\mu$ и  $<_\mu$ атомарни формули. 
\newline
Броят на точки в модела са $N = | \; W \; |$. Нека подредим точките $p_0, p_1, ..., p_N$. 
В такъв случай системата ще има N различни променливи $X_0, X_1, ..., X_N$, където $\forall i < N: X_i$ е съпоставена на точка $p_i$.
\end{frame}

\begin{frame}\frametitle{Построяване на система линейни неравенства}
	Нека x и y са два терма, тогава формулата $\le_\mu(x, y)$ се преобразува в:
	\begin{align*}
		\sum_{i: p_i \in \vE(x)} X_i \leq \sum_{j: p_j \in \vE(y)} X_j
	\end{align*}

			Това преобразуване може да се опрости до:
			\begin{align*}
				\sum_{i: p_i \in \vE(x) \setminus \vE(y)} X_i \leq \sum_{j: p_j \in \vE(y) \setminus \vE(x)} X_j
			\end{align*}
\end{frame}

\begin{frame}\frametitle{Построяване на система линейни неравенства}
			Нека x и y са два терма, тогава формулата $<_\mu(x, y)$ се преобразува в:
			\begin{align*}
				\sum_{i: p_i \in \vE(x)} X_i < \sum_{j: p_j \in \vE(y)} X_j
			\end{align*}

			Това преобразуване може да се опрости до:
			\begin{align*}
				\sum_{i: p_i \in \vE(x) \setminus \vE(y)} X_i < \sum_{j: p_j \in \vE(y) \setminus \vE(x)} X_j
			\end{align*}
\end{frame}

\begin{frame}\frametitle{Построяване на система линейни неравенства}
		\begin{defn}
			Нека $M = (W, R, \vE)$ е модел. Нека $\mathscr{S}$ е система от линейни неравенства дефинирана с:
			\begin{itemize}
				\item неравенство за всяка $\le_\mu$ формула
				\item неравенство за всяка $<_\mu$ формула
				\item неравенство $0 < X_i$ за всяко i: $0 \le i < N$  
			\end{itemize}
			Казваме, че системата $\mathscr{S}$ е валидна, ако тя има решение.
		\end{defn}
\end{frame}

\begin{frame}\frametitle{Изпълнимост в контактна логика с мярка}
С добавянето на атомарните формули с мярка се променят клоните на таблото. В тях се появяват и атомарни формули с мярка маркирани като вярни и невярни.
\begin{itemize}
	\item $\signT C(a, b)$
	\item $\signF C(a, b)$ 
	\item $\signT t = 0$
	\item $\signF t = 0$
	\item $\signT m_p$
	\item $\signF m_p$
\end{itemize}
\end{frame}

\begin{frame}\frametitle{Конюнктивен табло клон}
		За олеснение можем да махнем $\signT \textit{ и } \signF$ и получаваме.
\begin{align*}
			\bigwedge_{i=1}^{I} \; C(a_i, b_i) \:\: & \wedge \:\: 
			\bigwedge_{k=1}^{K} \; \neg C(e_k, f_k) \:\: \wedge \:\: \\
			\bigwedge_{j=1}^{J} \; d_j \neq 0 \:\: & \wedge \:\:
			\bigwedge_{l=1}^{L} \; g_l = 0 \:\: \wedge \:\:\\
			\bigwedge_{p=1}^{P} \; m_p \le_\mu n_p \:\: &\wedge \:\:
			\bigwedge_{q=1}^{Q} \; u_q <_\mu v_q \:\:
\end{align*}%
\end{frame}

\begin{frame}\frametitle{Изпълнимост в контактна логика с мярка}
	\begin{defn}%{Valid modal point}
	Нека B е клон в таблото. Казваме, че модалната точка p е валидна в B, когато:
	\begin{itemize}
		\item $t = 0 \in B:  \vBool_{p}(t) = \textbf{лъжа}$
		\item $\neg C(e, f) \in B: \vBool_{p}(e) = \textbf{лъжа} \textit{ или } \vBool_{p}(f) = \textbf{лъжа}$
	\end{itemize}
	\end{defn}

	\begin{defn}
		Нека B е клон в таблото и нека p и q са две валидни модални точки. Казваме, че $\langle p, q \rangle$ е валидна релация, когато:
		\begin{align*}
				\neg C(e, f) \in B: &(\vBool_{p}(e) = \textbf{лъжа} \textit{ или } \vBool_{q}(f) = \textbf{лъжа}) \textit{ и } \\
				 &(\vBool_{p}(f) = \textbf{лъжа} \textit{ или } \vBool_{q}(e) = \textbf{лъжа})
		\end{align*}
	\end{defn}
\end{frame}

\begin{frame}\frametitle{Изпълнимост в контактна логика с мярка}
	\begin{defn}
Нека B е клон в таблото и нека W е множество от валидни модални точки в B. Дефинираме модел $\mathcal{M} = (W, R, v)$ в B, където:
		\begin{align*}
				\vE(t) = \{ p \; | \; p \in W \textit{ и } &\vBool_{p}(t) =\; \textit{истина} \}, \\
				\textit{ където t е терм } & \textit{от атомарните формули в B}
		\end{align*}
		\begin{align*}
				R = \{ \langle p, q \rangle\; | \; p, q \in W \textit{ и } \langle p, q \rangle \textit{ е валидна релация}\}
		\end{align*}
	\end{defn}
\end{frame}

\begin{frame}\frametitle{Изпълнимост в контактна логика с мярка}
\begin{lema}[Невъзможни подмножествени модели]
Нека $\mathcal{M} = (W, R, \vE)$ е модел, където W е множество от валидни модални точки.
Нека $\mathcal{M'} = (W', R', \vE')$ е модел, където $W' \subseteq W, R' \subseteq R$, тогава:
	\begin{enumerate}
		\item $\mathcal{M} \not\models t \neq 0 \implies \mathcal{M'} \not\models t \neq 0$
		\item $\mathcal{M} \not\models C(a,b) \implies \mathcal{M'} \not\models C(a,b)$
	\end{enumerate}
	\end{lema}
\end{frame}

\begin{frame}\frametitle{Изпълнимост в контактна логика с мярка}
	\begin{lema}[Дедукция на променливите]
		Нека $\mathcal{M} = (W, R, \vE)$ е модел, където W е множество от валидни точки.
		Нека $\mathcal{M'} = (W', R', \vE')$ е подмодел на $\mathcal{M}$:
	% NOTE kaji che R' e sechenie na R s W2
		\begin{align*}
			\vE'(t) =  \vE(t) \cap W'
		\end{align*}
	\end{lema}
\end{frame}

\subsection{Алгоритъм за построяване на модел с мярка}
\begin{frame}\frametitle{Алгоритъм за построяване на модел с мярка}
	вход: $\varphi$ формула  \\
	изход: 
\begin{itemize}
	\item Не е изпълнима, ако $\neg \exists \mathcal{M}: \mathcal{M} \models \varphi$
	\item Модел $\mathcal{M}$, за който $\mathcal{M} \models \varphi$
\end{itemize}
\end{frame}

\begin{frame}
Стъпка 0:
\newline
Възможно е да може да се построи модел за всеки клон на таблото, затова ще проверяваме всеки клон докато не намерим модел.
\newline
Следващите стъпки работят върху един такъв клон.
\newline
\newline
Стъпка 1:
\newline
Генерираме модел $\mathcal{M}$ от всички валидни модални точки W в~B.
\end{frame}

\begin{frame}
Стъпка 2:
\newline
Нека $\mathbb{P} = \mathcal{P}(W) \setminus \{\emptyset\}$. Нека разгледаме $W' \in \mathbb{P}$ започвайки от подмножествата с най-много елементи.
			W' създава модела $\mathcal{M'}$ по лемата за дедукция на променливите.
			Проверяваме дали $\mathcal{M'}$ изпълнява неравно на нула термовете и контактите от B:
\newline
\newline
Стъпка 2.а:
\newline
 Ако са изпълними, тогава ако системата от линейни неравенства от B и $\mathcal{M'}$ има решение, то тогава валиден модел е построен, иначе W' се премахва от $\mathbb{P}$
\newline
\newline
Стъпка 2.б:
\newline
Ако не са изпълними, тогава по лемата за невъзможните подмножествени модели всички подмножества на W' се премахват от $\mathbb{P}$.
\end{frame}

\begin{frame}
Стъпка 3:
\newline
Ако W' не е намерено в стъпка 2., тогава не съществува модел с мярка, който да удовлетворява клона B.
\newline
\newline
Край:
\newline
Ако модел с мярка не е намерен за всеки от клоновете в таблото, то тогава $\neg \exists \mathcal{M}: \mathcal{M} \models \varphi$ 

\end{frame}

\begin{frame}\frametitle{Имплементация}
\begin{itemize}
	\item FLEX + BISON за построяване на формулата в AST (абстрактно синтактично дърво)
	\item Търсене на последователни атомарни клонове с табло метода
	\item Kiwi библиотека за смятане на система линейни неравенства
	\item Генериране на модел с мярка
	\item Уеб приложение за извикване на генерирането на модела и визуализиране на самия
\end{itemize}
 https://github.com/Anton94/modal\_logic\_formula\_prover
\end{frame}

\begin{frame}\frametitle{Демо}
http://logic.fmi.uni-sofia.bg/theses/Dudov\_Stoev/
\end{frame}

\begin{frame}
Благодаря за вниманието.
\end{frame}

\end{document}
























