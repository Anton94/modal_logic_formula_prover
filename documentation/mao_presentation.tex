% This text is proprietary.
% It's a part of presentation made by myself.
% It may not used commercial.
% The noncommercial use such as private and study is free
% Dec 2007
% Author: Sascha Frank 
% University Freiburg 
% www.informatik.uni-freiburg.de/~frank/
%
% 
\documentclass{beamer}
\usepackage[utf8]{inputenc}
\usepackage[T2A,T1]{fontenc}
\usepackage[english, bulgarian]{babel}
\setbeamertemplate{navigation symbols}{}

\usepackage{mathtools}
\usepackage{amsmath}
\usepackage{amssymb}
\usepackage{amsthm}
\usepackage{systeme}
\usepackage{newtxtext}
\usepackage{newtxmath}
\usepackage{listings}
\usepackage{xcolor}
\usepackage{array}
\usepackage{makecell}
\usepackage{booktabs}% http://ctan.org/pkg/booktabs
\usepackage{float}
\usepackage{hyperref}
\usepackage{tikz}
\usepackage{multicol}
\usepackage{graphicx}

\usetikzlibrary{arrows.meta}
\graphicspath{ {./image/} }

\restylefloat{table}
\newcommand{\tabitem}{~~\llap{\textbullet}~~}

\newtheorem{defn}{Дефиниция}[section]

\newcommand{\curvedE}{\mathscr{e}}
\newcommand{\vVar}{\mathscr{v}_{var}}
\newcommand{\Var}{\mathbb{V}ar}
\newcommand{\vE}{\mathscr{v}}
\newcommand{\vBool}{\xi}
\newcommand{\Ts}{\mathcal{T}_s}
\newcommand{\signT}{\mathbb{T}}
\newcommand{\signF}{\mathbb{F}}
\newcommand{\BranchConjunction}{%
\begin{align*}
			\bigwedge_{i=1}^{I} \; C(a_i, b_i) \:\: & \wedge \:\: 
			\bigwedge_{k=1}^{K} \; \neg C(e_k, f_k) \:\: \wedge \:\: \\
			\bigwedge_{j=1}^{J} \; d_j \neq 0 \:\: & \wedge \:\:
			\bigwedge_{l=1}^{L} \; g_l = 0 \:\:
\end{align*}%
}
\newcommand{\BranchConjunctionWithMeasure}{%
\begin{align*}
			\bigwedge_{i=1}^{I} \; C(a_i, b_i) \:\: & \wedge \:\: 
			\bigwedge_{k=1}^{K} \; \neg C(e_k, f_k) \:\: \wedge \:\: \\
			\bigwedge_{j=1}^{J} \; d_j \neq 0 \:\: & \wedge \:\:
			\bigwedge_{l=1}^{L} \; g_l = 0 \:\: \wedge \:\:\\
			\bigwedge_{p=1}^{P} \; m_p \le_\mu n_p \:\: &\wedge \:\:
			\bigwedge_{q=1}^{Q} \; u_q <_\mu v_q \:\:
\end{align*}%
}

\lstset { %
    literate={~} {$\sim$}{1},
    language=C++,
    backgroundcolor=\color{black!5}, % set backgroundcolor
    basicstyle=\footnotesize,% basic font setting
}



\usetheme{Montpellier}

\beamersetuncovermixins{\opaqueness<1>{25}}{\opaqueness<2->{15}}
\begin{document}
\title{Уеб система за минималната контактна логика с мярка}  
\author{Стоев Мартин}
\date{\today} 
\begin{frame}
\titlepage
\end{frame}

%\begin{frame}\frametitle{Съдържание}\tableofcontents\end{frame} 


\section{Въведение} 
\begin{frame}\frametitle{Въведение}
\begin{itemize}
	\item Защо модални логики ?
	\item Каква е целта на тази дипломна работа ?
	\item Теоретична част
	\item Практическа част
\end{itemize}
\end{frame}

\section{Съдържание} 
\begin{frame}\frametitle{Съдържание}
\begin{itemize}
	\item Табло Метод
	\item Минимална контактна логика
	\item Изпълнимост в минималната контактна логика
	\item Минимална контактна логика с мярка
	\item Изпълнимост в минималната контактна логика с мярка
\end{itemize}
\end{frame}

\section{Табло Метод}
\begin{frame}\frametitle{Табло Метод}
\begin{itemize}
	\item Табло метода като процедура за опровергаване на формули
	\item Табло Метод в Пропозиционалната Логика
\end{itemize}
\end{frame}

\subsection{Табло Метод в Пропозиционалната Логика}
\begin{frame}\frametitle{Табло Метод в Пропозиционалната Логика}
Маркиране на валидността на формула $\varphi$
\begin{itemize}
	\item $\signT \varphi$ - маркиране на формулата $\varphi$ за валидна
	\item $\signF \varphi$ - маркиране на формулата $\varphi$ за невалидна
\end{itemize}
\vspace{20px}
Стъпки на табло метода
\begin{itemize}
	\item Разшираване на табло метода
	\item Намиране на противоречия
\end{itemize}
\end{frame}

\begin{frame}\frametitle{Правила}
Негиране
\vspace{20px}
		\noindent\begin{minipage}{.5\linewidth}
		\begin{align*}
			\begin{array}{rl}
				& \signT(\neg \varphi), X \\
				      \cline{2-2}
				      & \signF(\varphi), X
			\end{array}
		\end{align*}
		\end{minipage}%
		\begin{minipage}{.5\linewidth}
		\begin{align*}
			\begin{array}{rl}
				& \signF(\neg \varphi), X \\
			      \cline{2-2}
			      & \signT(\varphi), X
			\end{array}
		\end{align*}
		\end{minipage}
Конюнкция
		\noindent\begin{minipage}{.5\linewidth}
		\begin{align*}
			\begin{array}{rl}
				& \signT(\varphi \: \wedge \: \psi), X \\
			      \cline{2-2}
			      & \signT\varphi, \signT\psi, X
			\end{array}
		\end{align*}
		\end{minipage}%
		\begin{minipage}{.5\linewidth}
		\begin{align*}
			\begin{array}{rl}
				& \:\:\: \signF(\varphi \: \wedge \: \psi), X \\
			      \cline{2-2}
			      & \signF\varphi, X \:\:\:\:\:\:\:\: \signF\psi, X
			\end{array}
		\end{align*}
		\end{minipage}
\end{frame}

\begin{frame}\frametitle{Правила}
Дизюнкция
\vspace{20px}
		\noindent\begin{minipage}{.5\linewidth}
		\begin{align*}
			\begin{array}{rl}
				& \:\:\: \signT(\varphi \: \vee \: \psi), X \\
			      \cline{2-2}
			      & \signT\varphi, X \:\:\:\:\:\:\:\: \signT\psi, X
			\end{array}
		\end{align*}
		\end{minipage}%
		\begin{minipage}{.5\linewidth}
		\begin{align*}
			\begin{array}{rl}
				& \signF(\varphi \: \vee \: \psi), X \\
			      \cline{2-2}
			      & \signF\varphi, \signF\psi, X
			\end{array}
		\end{align*}
		\end{minipage}
Импликация
		\noindent\begin{minipage}{.5\linewidth}
		\begin{align*}
			\begin{array}{rl}
				& \:\:\: \signT(\varphi \: \rightarrow \: \psi), X \\
			      \cline{2-2}
			      & \signF\varphi, X \:\:\:\:\:\:\:\: \signT\psi, X
			\end{array}
		\end{align*}
		\end{minipage}%
		\begin{minipage}{.5\linewidth}
		\begin{align*}
			\begin{array}{rl}
				& \signF(\varphi \: \rightarrow \: \psi), X \\
			      \cline{2-2}
			      & \signT\varphi, \signF\psi, X
			\end{array}
		\end{align*}
		\end{minipage}
\end{frame}

\begin{frame}\frametitle{Правила}
Еквивалентност
		\noindent\begin{minipage}{.5\linewidth}
		\begin{align*}
			\begin{array}{rl}
				& \:\:\:\:\:\:\:\: \signT(\varphi \: \leftrightarrow \: \psi), X \\
			      \cline{2-2}
			      & \signT\varphi, \signT\psi, X \:\:\:\:\:\:\:\: \signF\varphi, \signF\psi, X
			\end{array}
		\end{align*}
		\end{minipage}%
		\begin{minipage}{.5\linewidth}
		\begin{align*}
			\begin{array}{rl}
				& \:\:\:\:\:\:\:\: \signF(\varphi \: \leftrightarrow \: \psi), X \\
			      \cline{2-2}
			      & \signT\varphi, \signF\psi, X \:\:\:\:\:\:\:\: \signF\varphi, \signT\psi, X
			\end{array}
		\end{align*}
		\end{minipage}
\end{frame}

\begin{frame}\frametitle{Малко дефиниции}

\begin{defn}[Затворен клон]
Когато в него има едновременно една и съща формула маркирана за валидна и за невалидна.
\end{defn}

\begin{defn}[Затворено табло]
Когато в таблото всички клонове са затворени.
\end{defn}
\end{frame}

\begin{frame}\frametitle{Малко дефиниции}
\begin{defn}[Атомарен клон]
Когато клона не може да се разширява повече.
\end{defn}

\begin{defn}[Атомарно табло]
Когато в таблото всички клонове са атомарни.
\end{defn}

\begin{defn}[Завършено табло]
Когато таблото е затворено или атомарно.
\end{defn}
\end{frame}

\begin{frame}\frametitle{Общовалидна формула}
	Проверяваме дали дадена формула $\varphi$ е общовалидна с следните стъпки:
\begin{enumerate}
	\item Маркираме $\varphi$ за невалидна, т.е.  $\signF\varphi$.
	\item Ползваме $\signF\varphi$ за начална формула на таблото.
	\item Разширяваме все докато таблото не е завършено.
	\item Ако таблото е затворено, то формулата $\varphi$ е общовалидна.
\end{enumerate}
\end{frame}

\begin{frame}\frametitle{Пример}
	\hspace{50px} 1. $\signF (X \rightarrow ((X \wedge \neg Y) \vee \neg X)$ \\
	\hspace{50px} 2. $\signT X, \signF ((X \wedge \neg Y) \vee \neg X)$ \\
	\hspace{50px} 3. $\signT X, \signF (X \wedge \neg Y), \signF \neg X)$ \\
	\hspace{50px} 4. $\signT X, \signF (X \wedge \neg Y), \signT X)$ \\
	\begin{tikzpicture}[scale = 0.8]
		\draw (-7,0) (5,1);
        \draw[thick] (-3.1,0) -- (-2.2,1) -- (-1.3,0);
    \end{tikzpicture} \\
	\hspace{70px} 4. $\signF X$ \hspace{5px} 5. $\signF \neg Y$\\
	\hspace{108px} 6. $\signT Y$
\end{frame}

\section{Минимална контактна логика}
\begin{frame}\frametitle{Минимална контактна логика}
\begin{enumerate}
	\item Синтаксис 
	\item Семантика
	\item Свойства
	\item Изпълнимост на формула
\end{enumerate}
\end{frame}

\subsection{Синтаксис}
\begin{frame}\frametitle{Синтаксис}
	\begin{itemize}
		\item W - цял свят
		\item $\emptyset$ - празен регион
		\item $\mathbb{V}ar$ - изброимо множество от променливи използвани в дадена формула
		\item Булеви константи за W и $\emptyset$, 1 и 0 съответно
	\end{itemize}
\end{frame}

\begin{frame}\frametitle{Булеви операции}
	\begin{itemize}
		\item $\sqcap$ за булево сечение
		\item $\sqcup$ за булево обединение
		\item * за отрицание
	\end{itemize}
\end{frame}

\begin{frame}\frametitle{Дефиниция за Терм}
Терма се дефинира индуктивно:
	\begin{itemize}
		\item Булевите константи са термове
		\item $p \in \mathbb{V}ar$ е терм
		\item Ако x е терм, то *x е също така терм
		\item Ако x и y са два терма, то $x \; \sigma \; y $ е също така терм,\\
				където $\sigma \: \in \: \{\sqcap, \sqcup\}$
	\end{itemize}
\end{frame}

\begin{frame}\frametitle{Атомарни формули}
	Пропозиционални константи: $\top$ and $\bot$ \\
	\vspace{10px}
	Пропозиционални операции: $\neg, \vee, \wedge, \rightarrow, \leftrightarrow$
	\vspace{10px}

	Нека a и b са два терма. То тогава 
	\begin{itemize}
		\item C(a, b)
		\item $a \le b$
	\end{itemize}
	са атомарни формули.
\end{frame}

\begin{frame}\frametitle{Дефиниция за Формула}
Формула се дефинира индуктивно:
		\begin{itemize}
			\item Всяка пропозиционална константа е формула
			\item Всяка атомарна формула е формула
			\item Ако $\varphi$ е формула, то $\neg{\varphi}$ е също така формула
			\item Ако $\varphi$ и $\psi$ са две формули, то $\varphi \: \sigma \: \psi $ е също така формула,\\
				където $\sigma \in \{\vee, \wedge, \rightarrow, \leftrightarrow\}$
		\end{itemize}
\end{frame}

\subsection{Семантика}
\begin{frame}\frametitle{Семантика}

\end{frame}

\subsection{Свойства}
\begin{frame}\frametitle{Синтаксис}

\end{frame}

\subsection{Изпълнимост в минималната контактна логика}
\begin{frame}\frametitle{мини}
асд
\end{frame}

\section{Минимална контактна логика с мярка}
\begin{frame}\frametitle{мини}
асд
\end{frame}

\subsection{Изпълнимост в минималната контактна логика с мярка}
\begin{frame}\frametitle{мини}
асд
\end{frame}

\section{Section no. 2} 
\subsection{Lists I}
\begin{frame}\frametitle{unnumbered lists}
\begin{itemize}
\item Introduction to  \LaTeX  
\item Course 2 
\item Termpapers and presentations with \LaTeX 
\item Beamer class
\end{itemize} 
\end{frame}

\begin{frame}\frametitle{lists with pause}
\begin{itemize}
\item Introduction to  \LaTeX \pause 
\item Course 2 \pause 
\item Termpapers and presentations with \LaTeX \pause 
\item Beamer class
\end{itemize} 
\end{frame}

\subsection{Lists II}
\begin{frame}\frametitle{numbered lists}
\begin{enumerate}
\item Introduction to  \LaTeX  
\item Course 2 
\item Termpapers and presentations with \LaTeX 
\item Beamer class
\end{enumerate}
\end{frame}

\begin{frame}\frametitle{numbered lists with pause}
\begin{enumerate}
\item Introduction to  \LaTeX \pause 
\item Course 2 \pause 
\item Termpapers and presentations with \LaTeX \pause 
\item Beamer class
\end{enumerate}
\end{frame}

\section{Section no.3} 
\subsection{Tables}
\begin{frame}\frametitle{Tables}
\begin{tabular}{|c|c|c|}
\hline
\textbf{Date} & \textbf{Instructor} & \textbf{Title} \\
\hline
WS 04/05 & Sascha Frank & First steps with  \LaTeX  \\
\hline
SS 05 & Sascha Frank & \LaTeX \ Course serial \\
\hline
\end{tabular}
\end{frame}


\begin{frame}\frametitle{Tables with pause}
\begin{tabular}{c c c}
A & B & C \\ 
\pause 
1 & 2 & 3 \\  
\pause 
A & B & C \\ 
\end{tabular} 
\end{frame}


\section{Section no. 4}
\subsection{blocs}
\begin{frame}\frametitle{blocs}

\begin{block}{title of the bloc}
bloc text
\end{block}

\begin{exampleblock}{title of the bloc}
bloc text
\end{exampleblock}


\begin{alertblock}{title of the bloc}
bloc text
\end{alertblock}
\end{frame}

\section{Section no. 5}
\subsection{split screen}

\begin{frame}\frametitle{splitting screen}
\begin{columns}
\begin{column}{5cm}
\begin{itemize}
\item Beamer 
\item Beamer Class 
\item Beamer Class Latex 
\end{itemize}
\end{column}
\begin{column}{5cm}
\begin{tabular}{|c|c|}
\hline
\textbf{Instructor} & \textbf{Title} \\
\hline
Sascha Frank &  \LaTeX \ Course 1 \\
\hline
Sascha Frank &  Course serial  \\
\hline
\end{tabular}
\end{column}
\end{columns}
\end{frame}

\subsection{Pictures} 
\begin{frame}\frametitle{pictures in latex beamer class}
\begin{figure}
\caption{show an example picture}
\end{figure}
\end{frame}

\subsection{joining picture and lists} 

\begin{frame}
\frametitle{pictures and lists in beamer class}
\begin{columns}
\begin{column}{5cm}
\begin{itemize}
\item<1-> subject 1
\item<3-> subject 2
\item<5-> subject 3
\end{itemize}
\vspace{3cm} 
\end{column}
\begin{column}{5cm}
asdsad
\end{column}
\end{columns}
\end{frame}


\subsection{pictures which need more space} 
\begin{frame}[plain]
\frametitle{plain, or a way to get more space}
\begin{figure}
\caption{show an example picture}
\end{figure}
\end{frame}



\end{document}