% This text is proprietary.
% It's a part of presentation made by myself.
% It may not used commercial.
% The noncommercial use such as private and study is free
% Dec 2007
% Author: Sascha Frank 
% University Freiburg 
% www.informatik.uni-freiburg.de/~frank/
%
% 
\documentclass{beamer}
\usepackage[utf8]{inputenc}
\usepackage[T2A,T1]{fontenc}
\usepackage[english, bulgarian]{babel}
\setbeamertemplate{navigation symbols}{}

\usepackage{mathtools}
\usepackage{amsmath}
\usepackage{amssymb}
\usepackage{amsthm}
\usepackage{systeme}
\usepackage{newtxtext}
\usepackage{newtxmath}
\usepackage{listings}
\usepackage{xcolor}
\usepackage{array}
\usepackage{makecell}
\usepackage{booktabs}% http://ctan.org/pkg/booktabs
\usepackage{float}
\usepackage{hyperref}
\usepackage{tikz}
\usepackage{multicol}
\usepackage{graphicx}

\usetikzlibrary{arrows.meta}
\graphicspath{ {./image/} }

\restylefloat{table}
\newcommand{\tabitem}{~~\llap{\textbullet}~~}

\newtheorem{defn}{Дефиниция}[section]
\newtheorem{axiom}{Аксиома}[section]
\newtheorem{lema}{Лема}[section]

\newcommand{\curvedE}{\mathscr{e}}
\newcommand{\vVar}{\mathscr{v}_{var}}
\newcommand{\Var}{\mathbb{V}ar}
\newcommand{\vE}{\mathscr{v}}
\newcommand{\vBool}{\xi}
\newcommand{\Ts}{\mathcal{T}_s}
\newcommand{\signT}{\mathbb{T}}
\newcommand{\signF}{\mathbb{F}}
\newcommand{\BranchConjunction}{%
\begin{align*}
			\bigwedge_{i=1}^{I} \; C(a_i, b_i) \:\: & \wedge \:\: 
			\bigwedge_{k=1}^{K} \; \neg C(e_k, f_k) \:\: \wedge \:\: \\
			\bigwedge_{j=1}^{J} \; d_j \neq 0 \:\: & \wedge \:\:
			\bigwedge_{l=1}^{L} \; g_l = 0 \:\:
\end{align*}%
}
\newcommand{\BranchConjunctionWithMeasure}{%
\begin{align*}
			\bigwedge_{i=1}^{I} \; C(a_i, b_i) \:\: & \wedge \:\: 
			\bigwedge_{k=1}^{K} \; \neg C(e_k, f_k) \:\: \wedge \:\: \\
			\bigwedge_{j=1}^{J} \; d_j \neq 0 \:\: & \wedge \:\:
			\bigwedge_{l=1}^{L} \; g_l = 0 \:\: \wedge \:\:\\
			\bigwedge_{p=1}^{P} \; m_p \le_\mu n_p \:\: &\wedge \:\:
			\bigwedge_{q=1}^{Q} \; u_q <_\mu v_q \:\:
\end{align*}%
}

\lstset { %
    literate={~} {$\sim$}{1},
    language=C++,
    backgroundcolor=\color{black!5}, % set backgroundcolor
    basicstyle=\footnotesize,% basic font setting
}



\usetheme{Montpellier}

\beamersetuncovermixins{\opaqueness<1>{25}}{\opaqueness<2->{15}}
\begin{document}
\title{Уеб система за минималната контактна логика с мярка}  
\author{Стоев Мартин}
\date{\today} 
\begin{frame}
\titlepage
\end{frame}

%\begin{frame}\frametitle{Съдържание}\tableofcontents\end{frame} 


\section{Въведение} 
\begin{frame}\frametitle{Въведение}
\begin{itemize}
	\item Защо модални логики ?
	\item Каква е целта на тази дипломна работа ?
	\item Теоретична част
	\item Практическа част
\end{itemize}
\end{frame}

\section{Съдържание} 
\begin{frame}\frametitle{Съдържание}
\begin{itemize}
	\item Табло метод
	\item Минимална контактна логика
	\item Изпълнимост в минималната контактна логика
	\item Минимална контактна логика с мярка
	\item Изпълнимост в минималната контактна логика с мярка
\end{itemize}
\end{frame}

\section{Табло метод}
\begin{frame}\frametitle{Табло метод}
\begin{itemize}
	\item Табло метод като процедура за опровергаване на формули
	\item Табло метод като процедура за построяване на модели
	\item Табло Метод в Пропозиционалната Логика
\end{itemize}
\end{frame}

\subsection{Табло метод в пропозиционалната логика}
\begin{frame}\frametitle{Табло метод в пропозиционалната логика}
Маркиране на валидността на формула $\varphi$
\begin{itemize}
	\item $\signT \varphi$ - маркиране на формулата $\varphi$ за валидна
	\item $\signF \varphi$ - маркиране на формулата $\varphi$ за невалидна
\end{itemize}
\vspace{20px}
Стъпки на табло метода
\begin{itemize}
	\item Разшираване на табло метода
	\item Намиране на противоречия
\end{itemize}
\end{frame}

\begin{frame}\frametitle{Правила}
Негиране
\vspace{20px}
		\noindent\begin{minipage}{.5\linewidth}
		\begin{align*}
			\begin{array}{rl}
				& \signT(\neg \varphi), X \\
				      \cline{2-2}
				      & \signF(\varphi), X
			\end{array}
		\end{align*}
		\end{minipage}%
		\begin{minipage}{.5\linewidth}
		\begin{align*}
			\begin{array}{rl}
				& \signF(\neg \varphi), X \\
			      \cline{2-2}
			      & \signT(\varphi), X
			\end{array}
		\end{align*}
		\end{minipage}
Конюнкция
		\noindent\begin{minipage}{.5\linewidth}
		\begin{align*}
			\begin{array}{rl}
				& \signT(\varphi \: \wedge \: \psi), X \\
			      \cline{2-2}
			      & \signT\varphi, \signT\psi, X
			\end{array}
		\end{align*}
		\end{minipage}%
		\begin{minipage}{.5\linewidth}
		\begin{align*}
			\begin{array}{rl}
				& \:\:\: \signF(\varphi \: \wedge \: \psi), X \\
			      \cline{2-2}
			      & \signF\varphi, X \:\:\:\:\:\:\:\: \signF\psi, X
			\end{array}
		\end{align*}
		\end{minipage}
\end{frame}

\begin{frame}\frametitle{Правила}
Дизюнкция
\vspace{20px}
		\noindent\begin{minipage}{.5\linewidth}
		\begin{align*}
			\begin{array}{rl}
				& \:\:\: \signT(\varphi \: \vee \: \psi), X \\
			      \cline{2-2}
			      & \signT\varphi, X \:\:\:\:\:\:\:\: \signT\psi, X
			\end{array}
		\end{align*}
		\end{minipage}%
		\begin{minipage}{.5\linewidth}
		\begin{align*}
			\begin{array}{rl}
				& \signF(\varphi \: \vee \: \psi), X \\
			      \cline{2-2}
			      & \signF\varphi, \signF\psi, X
			\end{array}
		\end{align*}
		\end{minipage}
Импликация
		\noindent\begin{minipage}{.5\linewidth}
		\begin{align*}
			\begin{array}{rl}
				& \:\:\: \signT(\varphi \: \rightarrow \: \psi), X \\
			      \cline{2-2}
			      & \signF\varphi, X \:\:\:\:\:\:\:\: \signT\psi, X
			\end{array}
		\end{align*}
		\end{minipage}%
		\begin{minipage}{.5\linewidth}
		\begin{align*}
			\begin{array}{rl}
				& \signF(\varphi \: \rightarrow \: \psi), X \\
			      \cline{2-2}
			      & \signT\varphi, \signF\psi, X
			\end{array}
		\end{align*}
		\end{minipage}
\end{frame}

\begin{frame}\frametitle{Правила}
Еквивалентност
		\noindent\begin{minipage}{.5\linewidth}
		\begin{align*}
			\begin{array}{rl}
				& \:\:\:\:\:\:\:\: \signT(\varphi \: \leftrightarrow \: \psi), X \\
			      \cline{2-2}
			      & \signT\varphi, \signT\psi, X \:\:\:\:\:\:\:\: \signF\varphi, \signF\psi, X
			\end{array}
		\end{align*}
		\end{minipage}%
		\begin{minipage}{.5\linewidth}
		\begin{align*}
			\begin{array}{rl}
				& \:\:\:\:\:\:\:\: \signF(\varphi \: \leftrightarrow \: \psi), X \\
			      \cline{2-2}
			      & \signT\varphi, \signF\psi, X \:\:\:\:\:\:\:\: \signF\varphi, \signT\psi, X
			\end{array}
		\end{align*}
		\end{minipage}
\end{frame}

\begin{frame}\frametitle{Дефиниции}

\begin{defn}[Затворен клон]
Когато в него има едновременно една и съща формула маркирана за валидна и за невалидна.
\end{defn}

\begin{defn}[Атомарен клон]
Когато клона не може да се разширява повече.
\end{defn}

\begin{defn}[Завършено табло]
Когато всеки клон в таблото е или затворен или атомарен.
\end{defn}
\end{frame}

\begin{frame}\frametitle{Общовалидна формула}
	Проверяваме дали дадена формула $\varphi$ е общовалидна с следните стъпки:
\begin{enumerate}
	\item Маркираме $\varphi$ за невалидна, т.е.  $\signF\varphi$.
	\item Ползваме $\signF\varphi$ за начална формула на таблото.
	\item Разширяваме все докато таблото не е завършено.
	\item Ако всички клонове на таблото са затворени, то формулата $\varphi$ е общовалидна.
\end{enumerate}
\end{frame}

\begin{frame}\frametitle{Пример}
	\hspace{50px} 1. $\signF (X \rightarrow ((X \wedge \neg Y) \vee \neg X)$ \\
	\hspace{50px} 2. $\signT X, \signF ((X \wedge \neg Y) \vee \neg X)$ \\
	\hspace{50px} 3. $\signT X, \signF (X \wedge \neg Y), \signF \neg X)$ \\
	\hspace{50px} 4. $\signT X, \signF (X \wedge \neg Y), \signT X)$ \\
	\begin{tikzpicture}[scale = 0.8]
		\draw (-7,0) (5,1);
        \draw[thick] (-3.1,0) -- (-2.2,1) -- (-1.3,0);
    \end{tikzpicture} \\
	\hspace{70px} 4. $\signF X$ \hspace{5px} 5. $\signF \neg Y$\\
	\hspace{108px} 6. $\signT Y$
\end{frame}

\section{Минимална контактна логика}
\begin{frame}\frametitle{Минимална контактна логика}
\begin{enumerate}
	\item Синтаксис 
	\item Семантика
	\item Свойства
	\item Изпълнимост на формула
\end{enumerate}
\end{frame}

\subsection{Синтаксис}
\begin{frame}\frametitle{Синтаксис}
	\begin{itemize}
		\item W - цял свят
		\item $\emptyset$ - празен регион
		\item $\mathbb{V}ar$ - изброимо множество от променливи използвани в дадена формула
		\item Булеви константи за W и $\emptyset$, 1 и 0 съответно
	\end{itemize}
\end{frame}

\begin{frame}\frametitle{Булеви операции}
	\begin{itemize}
		\item $\sqcap$ за булево сечение
		\item $\sqcup$ за булево обединение
		\item * за отрицание
	\end{itemize}
\end{frame}

\begin{frame}\frametitle{Дефиниция за терм}
Терма се дефинира индуктивно:
	\begin{itemize}
		\item Булевите константи са термове
		\item $p \in \mathbb{V}ar$ е терм
		\item Ако x е терм, то *x е също така терм
		\item Ако x и y са два терма, то $x \; \sigma \; y $ е също така терм,\\
				където $\sigma \: \in \: \{\sqcap, \sqcup\}$
	\end{itemize}
\end{frame}

\begin{frame}\frametitle{Атомарни формули}
	Пропозиционални константи: $\top$ and $\bot$ \\
	\vspace{10px}
	Пропозиционални операции: $\neg, \vee, \wedge, \rightarrow, \leftrightarrow$
	\vspace{10px}

	Нека a и b са два терма. То тогава 
	\begin{itemize}
		\item C(a, b)
		\item $a \le b$
	\end{itemize}
	са атомарни формули.
\end{frame}

\begin{frame}\frametitle{Дефиниция за формула}
Формула се дефинира индуктивно:
		\begin{itemize}
			\item Всяка пропозиционална константа е формула
			\item Всяка атомарна формула е формула
			\item Ако $\varphi$ е формула, то $\neg{\varphi}$ е също така формула
			\item Ако $\varphi$ и $\psi$ са две формули, то $\varphi \: \sigma \: \psi $ е също така формула,\\
				където $\sigma \in \{\vee, \wedge, \rightarrow, \leftrightarrow\}$
		\end{itemize}
\end{frame}

\begin{frame}\frametitle{Съкращения}
		\begin{itemize}
		\item $a = b,\:\textit{когато}\:\: (a \le b) \land (b \le a)$
		\item $a \neq b,\:\textit{когато}\:\: \neg (a = b) $
		\item $a \nleq b,\:\textit{когато}\:\: \neg (a \le b)$
		\end{itemize}
\end{frame}

\subsection{Семантика}
\begin{frame}\frametitle{Семантика}
	Релационна система се дефинира като $\mathcal{F} = (W, R)$, където $W \neq \emptyset$. $\mathcal{F}$ наричаме фрейм.
	\newline
	\newline
	Булева оценка на променлива означаваме с  $\mathscr{v}$ и дефинираме като:
		\begin{itemize}
			\item $\mathscr{v}(0) = \emptyset$
			\item $\mathscr{v}(1) = W$
			\item $\mathscr{v}(a \sqcap b) = \mathscr{v}(a) \cap \mathscr{v}(b)$
			\item $\mathscr{v}(a \sqcup b) = \mathscr{v}(a) \cup \mathscr{v}(b)$
			\item $\mathscr{v}(a*) = W \backslash  \mathscr{v}(a)$
		\end{itemize}
	
\end{frame}

\subsection{Семантика}
\begin{frame}\frametitle{Модел}
	Наредената n-торка $\mathcal{M} = (\mathcal{F}, \mathscr{v})$ наричаме модел. Дефинираме изпълнимост на дадена формула в $\mathcal{M}$ като:
		\begin{itemize}
			\item $\mathcal{M} \not\models \bot$ 
			\item $\mathcal{M} \models \top$
			\item $\mathcal{M} \models aCb \text{ когато } (\exists x \in \mathscr{v}(a)), (\exists y \in \mathscr{v}(b)) (xRy)$
			\item $\mathcal{M} \models a \leq b \text{ когато } \mathscr{v}(a) \subseteq \mathscr{v}(b)$
			\item $\mathcal{M} \models \neg \varphi \text{ когато } \mathcal{M} \not\models \varphi$
			\item $\mathcal{M} \models a \; \vee \; b \text{ когато } \mathcal{M} \models a \: \textit{ или } \: \mathcal{M} \models b$
			\item $\mathcal{M} \models a \; \wedge \; b \text{ когато } \mathcal{M} \models a \: \textit{ и } \: \mathcal{M} \models b$
		\end{itemize}
	
\end{frame}

\subsection{Свойства}
\begin{frame}\frametitle{Свойства}

\begin{axiom}[Рефлексивност на контакта]
Нека b е терм, тогава:
	\begin{equation*}
		b \neq 0 \implies bCb
	\end{equation*}
\end{axiom}

\begin{axiom}[Симетрия на контакта]
Нека a и b са два терма, тогава:
	\begin{equation*}
		aCb \iff bCa
	\end{equation*}
\end{axiom}

\begin{lema}[Еквивалентност на термове]
Нека a и b са два терма и $\mathscr{v}$ е оценка , тогава:
	\begin{equation*}
		a = b \implies \mathscr{v}(a) = \mathscr{v}(b)
	\end{equation*}
\end{lema}

\end{frame}
\begin{frame}\frametitle{Свойства}
\begin{lema}[Нулева формула]
Нека a и b са два терма, тогава:
	\begin{equation*}
		a \le b \iff a \sqcap b* = \emptyset
	\end{equation*}
\end{lema}

\begin{lema}[Не-нулева формула]
Нека a и b са два терма, тогава:
	\begin{equation*}
		\neg(a \le b) \iff a \sqcap b* \neq \emptyset
	\end{equation*}
\end{lema}

\begin{lema}[Монотоност на контакта]
Нека a и b са два терма, тогава:
	\begin{equation*}
		aCb \land a \leq a' \land b \le b' \implies a'Cb'
	\end{equation*}
\end{lema}

\end{frame}
\begin{frame}\frametitle{Свойства}
\begin{lema}[Дистрибутивност на контакта]
Нека a и b са два терма, тогава:
	\begin{equation*}
		aC(b \sqcup c) \iff aCb \lor aCc
	\end{equation*}
\end{lema}

\begin{lema}[Тривиални свойства]
Нека a, b и c са три терма и $\varphi$ и $\psi$ са две формули, тогава:
	\begin{itemize}
		\item $\varphi \land T \Longrightarrow \varphi$,\;\; $T \land \varphi \Longrightarrow \varphi$,\;\;
		      $\varphi \land F \Longrightarrow F$,\;\; $F \land \varphi \Longrightarrow F$
		\item $\varphi \lor T \Longrightarrow T$,\;\; $T \lor \varphi \Longrightarrow T$,\;\;
		      $\varphi \lor F \Longrightarrow \varphi$,\;\; $F \lor \varphi \Longrightarrow \varphi$,
		\item $a \sqcap 1 \Longrightarrow a$,\;\; $1 \sqcap a \Longrightarrow a$,\;\;
		      $a \sqcap 0 \Longrightarrow 0$,\;\; $0 \sqcap a \Longrightarrow 0$,
		\item $a \sqcup 1 \Longrightarrow 1$,\;\; $1 \sqcup a \Longrightarrow 1$,\;\;
		      $a \sqcup 0 \Longrightarrow a$,\;\; $0 \sqcup a \Longrightarrow a$,
	\end{itemize}
\end{lema}
\end{frame}
\begin{frame}\frametitle{Тривиални свойства, продължение}
	\begin{itemize}
		\item $(a \sqcup b)Cc \iff aCc \lor bCc$
		\item $(a \sqcup b) \leq c \iff a \leq c \land b \leq c$
		\item $aCb \Longrightarrow a \neq 0 \land b \neq 0$
		\item $a \sqcap b \neq 0 \Longrightarrow aCb$
		\item $a = 0 \lor b = 0 \Longrightarrow \neg(aCb)$
		\item $0 \leq a \Longrightarrow T$
		\item $a \leq 1 \Longrightarrow T$
		\item $0C0 \Longrightarrow F$
		\item $aC0 \Longrightarrow F$
		\item $1C1 \Longrightarrow T$
		\item $aC1 \Longrightarrow a \neq 0$
		\item $a \neq 0 \Longrightarrow aCa$
	\end{itemize}
\end{frame}


\subsection{Изпълнимост в минималната контактна логика}
\begin{frame}\frametitle{Изпълнимост в минималната контактна логика}
За да проверим дали дадена формула $\varphi$ е изпълнима в минималната контактна логика трябва да построим модел $\mathcal{M}$
за да бъде вярно $\mathcal{M} \models \varphi$.
\newline
\newline
За да опростим формулата, ползваме табло метода и вместо да строим модел за първоначалната формула, строим модел само за 
атомарните клонове в таблото.
\newline
Намирането на един такъв модел $\mathcal{M}$ който изпълнява атомарните формули в клона е достатъчно, и няма нужда да се разглежда последващите клонове.
\end{frame}

\begin{frame}\frametitle{Конюнктивен табло клон}
Един атомарен клон се състои от маркирани формули. Дадената формула е изъплнима точно тогава, когато всички атомарни формули от таблото са изпълними.
\newline
\newline
		Нека $\varphi$ е формула и $\mathcal{T}$ е таблото от $\varphi$, то за означаване на конюктивен табло клон ще ползваме следното:
		\begin{align*}
				B = \{\signT C(a_i, b_i) \mid i \in \{1, \ldots, I\} \} &\cup 
					\{\signF C(e_k, f_k) \mid k \in \{1, \ldots, K\} \} \cup \\
					\{\signF d_j = 0 \mid j \in \{1, \ldots, J\} \} &\cup 
					\{\signT g_l = 0 \mid l \in \{1, \ldots, L\} \}
		\end{align*}
\end{frame}

\begin{frame}\frametitle{Конюнктивен табло клон}
		За олеснение можем да махнем $\signT \textit{ и } \signF$ и получаваме.
\begin{align*}
			\bigwedge_{i=1}^{I} \; C(a_i, b_i) \:\: & \wedge \:\: 
			\bigwedge_{k=1}^{K} \; \neg C(e_k, f_k) \:\: \wedge \:\: \\
			\bigwedge_{j=1}^{J} \; d_j \neq 0 \:\: & \wedge \:\:
			\bigwedge_{l=1}^{L} \; g_l = 0 \:\:
\end{align*}%
\end{frame}

\begin{frame}\frametitle{Конюнктивен табло клон}
\begin{defn}[Множеството на всички променливи]
	С $\Var_B$ ще ознчаваме множеството от всички променливи използвани в дадена формула.
\end{defn}

\begin{defn}[Оценка на променливи]
	С $\curvedE$ ще ознчаваме функцията която за всяка оценка от $\Var_B$ дава истина или лажа.
		\begin{align*}
			\curvedE : \Var_B \rightarrow \{ \textbf{лъжа}, \textbf{истина}\}
		\end{align*}
\end{defn}
\end{frame}

\begin{frame}\frametitle{Конюнктивен табло клон}
\begin{defn}[Булева оценка]
		Нека $\curvedE$ е оценка на променливи и $\Ts$ е множеството от всички термове, тогава 
		функцията $\vBool_{\curvedE} : \Ts \rightarrow \{ \textbf{лъжа}, \textbf{истина}\}$ ще наричаме булева оценка, която се дефинира по следния начин:
		\begin{itemize}
			\item $\vBool_{\curvedE}(0) = \textbf{лъжа}$
			\item $\vBool_{\curvedE}(1) = \textbf{истина}$
			\item $\vBool_{\curvedE}(p) = \curvedE(p), \textit{ където } p \in \Var_B$
			\item $\vBool_{\curvedE}(a \sqcap b) = \vBool_{\curvedE}(a) \textit{ и } \vBool_{\curvedE}(b)$
			\item $\vBool_{\curvedE}(a \sqcup b) = \vBool_{\curvedE}(a) \textit{ или } \vBool_{\curvedE}(b)$
			\item $\vBool_{\curvedE}(a*) = \textit{не } \vBool_{\curvedE}(a)$
		\end{itemize}
\end{defn}
\end{frame}

\begin{frame}\frametitle{Стъпки за построяване на модел}
	Ще казваме, че модална точка $\curvedE$ е построена за терма a, когато:
		\begin{align*}
			\vBool_{\curvedE}(a) = \textit{ истина }
		\end{align*}
	Ще групираме атомарните формули на такива за които е необходимо съществуването на модална точка и на такива за които не е.
\newline
	Следните атомарни формули се нуждаят от същестуването на поне една модална точка:
		\begin{itemize}
			\item $C(a_i, b_i), \textit{ for } i < I$
			\item $d_j \neq 0, \textit{ for } j < J$
		\end{itemize}
\end{frame}

\begin{frame}\frametitle{Построяване на модални точки за контактите}
	За всеки контакт $C(a, b) \in B$ ще построим по две модални точки, такива, че:
		\begin{itemize}
			\item $\vBool_{p}(a) = \textbf{истина}$
			\item $\vBool_{q}(b) = \textbf{истина}$
		\end{itemize}
\end{frame}

\begin{frame}\frametitle{Построяване на модални точки за контактите}
	Ново генерираните модални точки трябва да удовлетворяват не-контактите, т.е. следното условие трябва да е изпълнено:
		\begin{align*}
			 \neg C(e, f) \in B: (\vBool_{p}(e) = \textbf{лъжа} & \textit{ или } \vBool_{q}(f) = \textbf{лъжа}) \; \textit{ и } \\
				(\vBool_{p}(f) = \textbf{лъжа} & \textit{ или } \vBool_{q}(e) = \textbf{лъжа}) \; \textit{ и } \\
				(\vBool_{p}(e) = \textbf{лъжа} & \textit{ или } \vBool_{p}(f) = \textbf{лъжа}) \; \textit{ и } \\
				(\vBool_{q}(e) = \textbf{лъжа} & \textit{ или } \vBool_{q}(f) = \textbf{лъжа})
		\end{align*}

	Също така за равно на нула термовете, следното условие трябва да е изпълнено:
	\begin{align*}
		t = 0 \in B: \vBool_{p}(t) = \textit{лъжа} \text{ и }  \vBool_{q}(t) = \textit{лъжа}
	\end{align*}
\end{frame}

\begin{frame}\frametitle{Построяване на модални точки за контактите}
	След успешно генерираните модални точки за двата терма, обогатяваме R със следните релации:
	\begin{itemize}
		\item pRp - рефлексивност на модалната точка p
		\item qRq - рефлексивност на модалната точка q
		\item pRq - симетричност между p и q
		\item qRp - симетричност между q и p
	\end{itemize}
\end{frame}

\begin{frame}\frametitle{Построяване на модални точки за контактите}
	\begin{center}
	\begin{tikzpicture}[scale = 1]
		\filldraw[color=black!60, fill=red!0, thick](-3,0) circle (0.5);
		\node at (-3,0) {p};
		\node at (-3.5,-0.7) {a};
		\draw[thick, ->] (-3.5,0) arc [
      	  		start angle=10,
		       end angle=335,
        		x radius=0.5cm,
        		y radius =0.2cm
    			] ;
		\draw[thick, <->](-2.5,0) -- (2.5,0);
		\filldraw[color=black!60, fill=red!0, thick](3,0) circle (0.5);
		\node at (3,0) {q};
		\node at (3.5,-0.7) {b};
		\draw[thick, ->] (3.5,0) arc [
      	  		start angle=180,
		      end angle=525,
        		x radius=0.5cm,
        		y radius =0.2cm
    			] ;
		\node[text width=10cm, anchor=west, right] at (-4.5,-2)
    		{Генерираните модални точки и техните релации за контакта C(a, b)};
      \end{tikzpicture}
	\end{center}
\end{frame}

\begin{frame}\frametitle{Построяване на модална точка за не-равен на нула терм}
	За всеки не-равен на нула терм $a \neq 0 \in B$ ще построим една модални точки, такава, че:
		\begin{itemize}
			\item $\vBool_{\curvedE}(a) = \textbf{истина}$
		\end{itemize}
	За равно на нула термовете, следното условие трябва да е изпълнено:
	\begin{align*}
		t = 0 \in B: \vBool_{\curvedE}(t) = \textit{лъжа}
	\end{align*}
	След успешно генериране на модалната точка, отново обогатяваме R със следната:
	\begin{itemize}
		\item $\curvedE R \curvedE$ - рефлексивност на модалната точка $\curvedE$
	\end{itemize}
\end{frame}

\begin{frame}\frametitle{Построяване на модална точка за не-равен на нула терм}
	\begin{center}
	\begin{tikzpicture}[scale = 1]
		\filldraw[color=black!60, fill=red!0, thick](-0,0) circle (0.5);
		\draw[thick, ->] (-0.5,0) arc [
      	  		start angle=10,
		       end angle=345,
        		x radius=0.5cm,
        		y radius =0.2cm
    			] ;
		\node at (-0,0) {$\curvedE$};
		\node at (-0.5,-0.7) {a};
		\node[text width=10cm, anchor=west, right] at (-4.5,-2)
    		{Генерираната модална точка и нейната релация за $a \neq 0$};
      \end{tikzpicture}
	\end{center} 
\end{frame}

\begin{frame}\frametitle{Построяване на модел}
	\begin{defn}
		Нека $\Ts$ е множеството от всички термове и нека $\mathcal{F}$ е фрейм създаден с построяване на модални точки за контактите и не-равно на нула термове.
		Тогава модалната оценка $\vE : \Ts \rightarrow \mathscr{P}(W)$ се дефинира рекурсивно, като:
		\begin{itemize}
			\item $\vE(0) = W$
			\item $\vE(1) = \emptyset$
			\item $\vE(p) = \{ \curvedE \; | \; \curvedE \in W \:\:\textit{и}\:\: \curvedE(p) = \textit{истина} \}$
			\item $\vE(a \sqcap b) = \vE(a) \cap \vE(b)$
			\item $\vE(a \sqcup b) = \vE(a) \cup \vE(b)$
			\item $\vE(a*) = W \setminus \vE(a)$
		\end{itemize}
	\end{defn}
\end{frame}

\begin{frame}\frametitle{Построяване на модел}
	\begin{lema}
		Нека a е терм и нека $\curvedE$ е оценка на променливи, от дефиницията на $\vBool$ и $\vE$, следва, че:
		\begin{align*}
			\vBool_{\curvedE}(a) = истина \leftrightarrow \curvedE \in \vE(a)
		\end{align*}
		В такъв случай, когато $\vBool_{\curvedE}(a) = \textbf{истина}$ ще казваме, че модална точка  $\curvedE$ е \textbf{валидна}.
	\end{lema}
\end{frame}

\section{Минимална контактна логика с мярка}
\begin{frame}\frametitle{Минимална контактна логика с мярка}
Минималната контактна логика с мярка е самата минимална контактна логика с добавена количествена мярка.
\newline
\newline
Мярката е функция която на даден на регион съпоставя положително реално число.
\begin{equation*}
	\mu : \mathscr{P}(W) \longrightarrow \mathbb{R}^+
\end{equation*}
Мярката между два терма представляваме с следната атомарна формула:
	\begin{equation*} 
		a \le_\mu b
	\end{equation*}
\end{frame}


\begin{frame}\frametitle{Семантика на минималната контактна логика с мярка}
	Модела на минималната контактна логика се разширява със следната индуктивна дефиниция за изпълнимост:
	\begin{itemize}
		\item $\mathcal{M} \models a \leq_\mu b \text{, когато } \mu(\mathscr{v}(a)) \le \mu(\mathscr{v}(b))$
	\end{itemize}

	\begin{lema}[Тривиални импликации за модели с мярка]
Нека a, b и c са три терма и нека $\varphi$ и $\psi$ са две формули, тогава:
	\begin{itemize}
		\item $0 \le_\mu a \Longrightarrow T$
		\item $a \le_\mu 1 \Longrightarrow T$
		\item $(a = 0) \iff (a \le_\mu 0)$
		\item $(a = 1) \iff (1 \le_\mu a)$
		\item $(a \le_\mu b) \lor (b \le_\mu a) \Longrightarrow T$
		\item $a_1 \le_\mu a_2 \land a_2 \le_\mu a_3 \Longrightarrow a_1 \le_\mu a_3$
	\end{itemize}
\end{lema}
\end{frame}

\begin{frame}\frametitle{Система линейни неравенства}
Формула с повече атомарни формули с мярка създава система линейни неравенства.
\newline
Системата линейни неравенства има следната структура:
		\[
			\syslineskipcoeff{1.50}
			\systeme*{\sum_{i^1} X_{i^1} <= \sum_{j^1} X_{j^1},
				\:\:.\:\:.\:\:.,
				\sum_{i^n} X_{i^n} <= \sum_{j^n} X_{j^n}, 
				\sum_{k^1} X_{k^1} > \sum_{l^1} X_{l^1},
				\:\:.\:\:.\:\:.,
				\sum_{k^m} X_{k^m} > \sum_{l^m} X_{l^m}
				}
		\]
\end{frame}

\begin{frame}\frametitle{Построяване на система линейни неравенства}
Нека $M = (W, R, \vE)$ е модел. Системата се построява с оценяване на термовете в $\le_\mu$ и  $<_\mu$ атомарни формули. 
\newline
Броят на точки в модела са $N = | \; W \; |$. Нека подредим точките $p_0, p_1, ..., p_N$. 
В такъв случай системата ще има N различни променливи $X_0, X_1, ..., X_N$, където $\forall i < N: X_i$ е съпоставена на точка $p_i$.
\end{frame}

\begin{frame}\frametitle{Построяване на система линейни неравенства}
\begin{defn}
	Нека x и y са два терма, тогава формулата $\le_\mu(x, y)$ се преобразува в:
	\begin{align*}
		\sum_{i: p_i \in \vE(x)} X_i \leq \sum_{j: p_j \in \vE(y)} X_j
	\end{align*}
\end{defn}

		\begin{lemma}
			Това преобразуване може да се опрости до:
			\begin{align*}
				\sum_{i: p_i \in \vE(x) \setminus \vE(y)} X_i \leq \sum_{j: p_j \in \vE(y) \setminus \vE(x)} X_j
			\end{align*}
		\end{lemma}
\end{frame}

\begin{frame}\frametitle{Построяване на система линейни неравенства}
		\begin{defn}
			Нека x и y са два терма, тогава формулата $<_\mu(x, y)$ се преобразува в:
			\begin{align*}
				\sum_{i: p_i \in \vE(x)} X_i < \sum_{j: p_j \in \vE(y)} X_j
			\end{align*}
		\end{defn}

		\begin{lemma}
			Това преобразуване може да се опрости до:
			\begin{align*}
				\sum_{i: p_i \in \vE(x) \setminus \vE(y)} X_i < \sum_{j: p_j \in \vE(y) \setminus \vE(x)} X_j
			\end{align*}
		\end{lemma}
\end{frame}

\begin{frame}\frametitle{Построяване на система линейни неравенства}
		\begin{defn}
			Нека $M = (W, R, \vE)$ е модел. Нека $\mathscr{S}$ е система от линейни неравенства дефинирана с:
			\begin{itemize}
				\item неравенство за всяка $\le_\mu$ формула
				\item неравенство за всяка $<_\mu$ формула
				\item неравенство $0 < X_i$ за всяко i: $0 \le i < N$  
			\end{itemize}
			Казваме, че системата $\mathscr{S}$ е валидна ако тя има решение.
		\end{defn}
\end{frame}

\begin{frame}\frametitle{Изпълнимост в минималната контактна логика с мярка}
С добавянето на атомарните формули с мярка се променя и таблото и конюктивия табло клон.
Нека $\varphi$ е формула и $\mathcal{T}$ е таблото от $\varphi$, тогава:
		\begin{align*}
				B = \{\signT C(a_i, b_i) \mid i \in \{1, \ldots, I\} \} &\cup 
					\{\signF C(e_k, f_k) \mid k \in \{1, \ldots, K\} \} \cup \\
					\{\signF d_j = 0 \mid j \in \{1, \ldots, J\} \} &\cup 
					\{\signT g_l = 0 \mid l \in \{1, \ldots, L\} \} \cup \\
					\{\signT m_p \le_\mu n_p \mid p \in \{1, \ldots, P\} \} &\cup
					\{\signF u_q \le_\mu v_q \mid q \in \{1, \ldots, Q\} \}
		\end{align*}
\end{frame}

\begin{frame}\frametitle{Конюнктивен табло клон}
		За олеснение можем да махнем $\signT \textit{ и } \signF$ и получаваме.
\begin{align*}
			\bigwedge_{i=1}^{I} \; C(a_i, b_i) \:\: & \wedge \:\: 
			\bigwedge_{k=1}^{K} \; \neg C(e_k, f_k) \:\: \wedge \:\: \\
			\bigwedge_{j=1}^{J} \; d_j \neq 0 \:\: & \wedge \:\:
			\bigwedge_{l=1}^{L} \; g_l = 0 \:\: \wedge \:\:\\
			\bigwedge_{p=1}^{P} \; m_p \le_\mu n_p \:\: &\wedge \:\:
			\bigwedge_{q=1}^{Q} \; u_q <_\mu v_q \:\:
\end{align*}%
\end{frame}

\begin{frame}\frametitle{Изпълнимост в минималната контактна логика с мярка}
	\begin{defn}%{Valid modal point}
	Нека B е конюктивен табло клон. Казваме, че модалната точка $\curvedE$ е валидна в B, когато:
	\begin{itemize}
		\item $t = 0 \in B:  \vBool_{\curvedE}(t) = \textbf{лъжа}$
		\item $\neg C(e, f) \in B: \vBool_{\curvedE}(e) = \textbf{лъжа} \textit{ или } \vBool_{\curvedE}(f) = \textbf{лъжа}$
	\end{itemize}
	\end{defn}

	\begin{defn}
		Нека B е конюктивен табло клон и нека x и y са две валидни модални точки. Казваме, че $\langle x, y \rangle$ е валдина релация, когато:
		\begin{align*}
				\neg C(e, f) \in B: (x \notin \vE(e) \textit{ или } y \notin \vE(f)) \textit{ и } (x \notin \vE(f) \textit{ или } y \notin \vE(e))
		\end{align*}
	\end{defn}
\end{frame}

\begin{frame}\frametitle{Изпълнимост в минималната контактна логика с мярка}
	\begin{defn}
Нека B е конюктивен табло клон и нека W е множество от валидни модални точки в B. Дефинираме модел $\mathcal{M} = (W, R, v)$ в B, където:
		\begin{align*}
				\vE(t) = \{ \curvedE \; | \; \curvedE \in W \textit{ и } &\vBool_{\curvedE}(t) =\; \textit{истина} \}, \\
				\textit{ където t е терм } & \textit{от атомарните формули в B}
		\end{align*}
		\begin{align*}
				R = \{ \langle x, y \rangle\; | \; x, y \in W \textit{ и } \langle x, y \rangle \textit{ е валидна релация}\}
		\end{align*}
	\end{defn}
\end{frame}

\begin{frame}\frametitle{Изпълнимост в минималната контактна логика с мярка}
\begin{lema}[Невъзможни подмножествени модели]
Нека $\mathcal{M} = (W, R, \vE)$ е модел, където W е множество от валидни модални точки.
Нека $\mathcal{M'} = (W', R', \vE')$ е модел, където $W' \subseteq W, R' \subseteq R$, тогава:
	\begin{enumerate}
		\item $\mathcal{M} \not\models t \neq 0 \implies \mathcal{M'} \not\models t \neq 0$
		\item $\mathcal{M} \not\models C(a,b) \implies \mathcal{M'} \not\models C(a,b)$
	\end{enumerate}
	\end{lema}
\end{frame}

\begin{frame}\frametitle{Изпълнимост в минималната контактна логика с мярка}
	\begin{lema}[Дедукция на променливите]
		Нека $\mathcal{M} = (W, R, \vE)$ е модел, където W е множество от валидни точки.
		Нека $\mathcal{M'} = (W', R', \vE')$ е модел, дефиниран от $W' \subseteq W$, тогава:
	% NOTE kaji che R' e sechenie na R s W2
		\begin{align*}
			\vE'(t) =  \vE(t) \cap W'
		\end{align*}
	\end{lema}
\end{frame}

\subsection{Алгоритъм за построяване на модел с мярка}
\begin{frame}\frametitle{Алгоритъм за построяване на модел с мярка}
	вход: $\varphi$ формула  \\
	изход: 
\begin{itemize}
	\item Не е изпълнима, ако $\neg \exists \mathcal{M}: \mathcal{M} \models \varphi$
	\item Модел $\mathcal{M}$, за който $\mathcal{M} \models \varphi$
\end{itemize}
\end{frame}

\begin{frame}
Стъпка 0:
\newline
Възможно е да може да се построи модел в всеки конюнктивен клон на таблото, затова ще проверяваме всеки клон докато не намерим модел.
\newline
Следващите стъпки работят върху един такъв клон.
\newline
\newline
Стъпка 1:
\newline
Генерираме модел $\mathcal{M}$ от валидни модални точки W в B.
\end{frame}

\begin{frame}
Стъпка 2:
\newline
Нека $\mathbb{P} = \mathcal{P}(W)$. Нека разгледаме $W' \in \mathbb{P}$ започвайки от подмножествата с най-много елементи.
			W' създава модела $\mathcal{M'}$ по лемата за дедукция на променливите.
			Проверяваме дали $\mathcal{M'}$ изпълнява не-равно на нула термовете и контактие от B:
\newline
\newline
Стъпка 2.а:
\newline
 Ако са изпълними, тогава ако системата от линейни неравенства от B и $\mathcal{M'}$ има решение, то тогава валиден модел е построен, иначе W' се премахва от $\mathbb{P}$
\newline
\newline
Стъпка 2.б:
\newline
Ако не са изпълними, тогава по лемата за невъзможните подмножествени модели всички подмножества на W' се премахват от $\mathbb{P}$.
\end{frame}

\begin{frame}
Стъпка 3:
\newline
Ако W' не е намерено в стъпка 2., тогава не съсществува модел с мярка който да удовлетворява конюнктивния клон B.
\newline
\newline
Край:
\newline
Ако модел с мярка не е намерен в всеки от конюнктивните клонове в таблото, то тогава $\neg \exists \mathcal{M}: \mathcal{M} \models \varphi$ 

\end{frame}

\begin{frame}
Благодаря ви. \\
Нека да пробваме самата програмка!
\end{frame}

\end{document}
























