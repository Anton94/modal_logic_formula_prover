\documentclass[14pt, aspectratio=169]{beamer}

% sudo apt-get install texlive-lang-cyrillic
\usepackage[T2A,T1]{fontenc}
\usepackage[utf8]{inputenc}
\usepackage[main=bulgarian,english]{babel}

\usepackage{forest}
\usepackage{mathtools}
\forestset{
    parse tree/.style={for tree={s sep=0.5em, minimum size=0.5em}}
}

\graphicspath{ {./image/} }

\usetheme{Boadilla}
\usecolortheme{beaver}

% Hide the navigation buttons at the bottom of each slide
\setbeamertemplate{navigation symbols}{}

% Lets the boolet points for next slides to be seen
%\setbeamercovered{transparent}

% Increase the size of title in the footer
\makeatletter
\defbeamertemplate*{footline}{Dan P theme}
{
  \leavevmode%
  \hbox{%
  \begin{beamercolorbox}[wd=.2\paperwidth,ht=2.25ex,dp=1ex,center]{author in head/foot}%
    \usebeamerfont{author in head/foot}\insertshortauthor\expandafter\beamer@ifempty\expandafter{\beamer@shortinstitute}{}{~~(\insertshortinstitute)}
  \end{beamercolorbox}%
  \begin{beamercolorbox}[wd=.5\paperwidth,ht=2.25ex,dp=1ex,center]{title in head/foot}%
    \usebeamerfont{title in head/foot}\insertshorttitle
  \end{beamercolorbox}%
  \begin{beamercolorbox}[wd=.3\paperwidth,ht=2.25ex,dp=1ex,right]{date in head/foot}%
    \usebeamerfont{date in head/foot}\insertshortdate{}\hspace*{2em}
\insertframenumber{} / \inserttotalframenumber\hspace*{2ex}
  \end{beamercolorbox}}%
  \vskip0pt%
}
\makeatother

% Teach babel the tags to translate them
\usepackage{etoolbox}
\patchcmd{\theorem}{Theorem}{Teorema}{}{}
\patchcmd{\corollary}{Corollary}{Следствие}{}{}
\patchcmd{\lemma}{Lemma}{Лема}{}{}
\patchcmd{\proposition}{Proposition}{Предложение}{}{}
\patchcmd{\axiom}{Axiom}{Аксиома}{}{}
\patchcmd{\example}{Example}{Пример}{}{}
\patchcmd{\definition}{Definition}{Дефиниция}{}{}
\patchcmd{\remark}{Remark}{Забележка}{}{}

\newcommand\ST{\mathbb{T}}
\newcommand\SF{\mathbb{F}}
\newcommand\SB{\mathbb{B}}
\newcommand\M{\mathcal{M}}
\newcommand\F{\mathcal{F}}
\newcommand\p{\mathcal{P}}
\newcommand\V{\mathcal{V}}
\newcommand\E{\mathcal{E}}
\newcommand\Tb{\mathbf{T}}
\newcommand\Vb{\mathbf{V}}

\logo{\includegraphics[width=1cm]{fmi_logo.png}}

\date[\today]{}

\title{\textsc{Уеб система за изпълнимост на свързаната контактна логика}}

%\subject{}

%\institute{}

\author[Антон Дудов]{
	\textsc{Факултет по математика и информатика} \\
	\textsc{Катедра по математическа логика и приложенията ѝ} \\ [8mm]
	\textsc{Антон Дудов} \\ [8mm]
	\small\textsc{Научен ръководител: проф. Тинко Тинчев}
}

\begin{document}

\begin{frame}
\titlepage
\end{frame}

\begin{frame}
\tableofcontents
\end{frame}

\section{Табло метод за класическа съждителна логика}

% Табло методът има няколко приложения.
% Може да се използва за формално доказване, че дадена съждителна формула е тавтология.
% Алгоритъм за търсене на модел, в който формулата е изпълнима.
\begin{frame}{Табло метод за класическа съждителна логика}
Приложения:
\begin{itemize}
	\item<1-> Доказване, че формула е тавтология
	\begin{description}
		\item<2-> $\phi = x \lor \neg x$
	\end{description}

	\item<3-> Алгоритъм за търсене на модел % Алгоритъм за търсене на модел, в който формулата е изпълнима
	\begin{description}
		\item<4-> $\psi = (x \land \neg x) \lor (\neg x \land y) \rightarrow x = F ,\; y = T$
	\end{description}
\end{itemize}
\end{frame}

% Първо, трябва ни синтактичен способ, с който да покажем, че дадена формула не е изпълнима и да правим анализ по формулата.
% Ще следваме описанието на табло метода от книгата - Handbook of tableau methods за табло методите. Там биват два вида табла - със или без знаци. Ще ползваме версията със знаци.
% Знаците, с които ще маркираме формулите са F и T.
% Формулите със знаци са изрази от вида FX и TX

\begin{frame}{Табло метод за класическа съждителна логика}
	Табло метод със знаци $\ST$ и $\SF$
	\begin{itemize}
		\item $\ST X$ - означава, че формулата X трябва да е true (в някой модел)
		\item $\SF X$ - аналогично, X трябва да е false
	\end{itemize}
\end{frame}

\begin{frame}{Табло метод - правила за разбиване на подформули}
	\begin{itemize}
	\LARGE
		\item<1-> $\frac{\ST \neg X}{\SF X}\;\;\;\;\;\;\;\;\;\;\;\;\;\;\;\;\;\;\;\;\;\;\ \frac{\SF \neg X}{\ST X}$
		\\ [8mm]
		\item<2-> $\frac{\ST X \land Y}{{\ST X \atop \ST Y}} \;\;\;\;\;\;\;\;\;\;\;\;\;\;\;\;\;\;\;\;\; \frac{\SF X \land Y}{\SF X \mid \SF Y}$
		\\ [8mm]
		\item<3-> $\frac{\ST X \lor Y}{\ST X \mid \ST Y} \;\;\;\;\;\;\;\;\;\;\;\;\;\;\;\;\;\;\;\ \frac{\SF X \lor Y}{{\SF X \atop \SF Y}}$
	\normalsize
	\end{itemize}
\end{frame}

\begin{frame}{Табло метод - правила за разбиване на подформули}
	\begin{itemize}
	\LARGE
		\item<1-> $\frac{\ST X \Rightarrow Y}{\SF X \mid \ST Y} \;\;\;\;\;\;\;\;\;\;\;\;\;\;\;\;\;\;\;\;\;\ \frac{\SF X \Rightarrow Y}{{\ST X \atop \SF Y}}$
		\\ [8mm]
		\item<2-> $\frac{\ST X \Leftrightarrow Y}{\left.{\ST X \atop \ST Y}\right\vert{\SF X \atop \SF Y}} \;\;\;\;\;\;\;\;\;\;\;\;\;\;\;\;\;\;\;\;\;\ \frac{\SF X \Leftrightarrow Y}{\left.{\ST X \atop \SF Y}\right\vert{\SF X \atop \ST Y}}$
	\normalsize
	\end{itemize}
\end{frame}

\begin{frame}{Табло метод - строене}
	\begin{center}
	\begin{forest}
		parse tree
		[$\ST ((x \land \neg x) \lor (\neg x \land y))$
			[$\ST (x \land \neg x)$ [$\ST x$ [$\ST \neg x$ [$\SF x$]]]]
			[$\ST (\neg x \land y)$ [$\ST \neg x$ [$\ST y$ [$\SF x$]]]]
		]
	\end{forest}
	\end{center}
\end{frame}

\begin{frame}{Табло метод - дефиниции}
	\begin{itemize}
		\item<1-> Клон се нарича \textbf{затворен}, ако съдържа противоречие.
		\item<2-> Клон се нарича \textbf{приключен}, ако всички формули в него са приложени, т.е. съдържа само променливи.
		\item<3-> Клон се нарича \textbf{отворен}, ако е приключен и не е затворен.
		\item<4-> \textbf{Затворено табло} е табло, на което всички клонове са затворени.
	\end{itemize}
\end{frame}

\begin{frame}{Табло метод - тавтология}
	\begin{lemma}
		Затворено табло за $\SF X$ е табло доказателство за X, т.е. Х е \textbf{тавтология}.
	\end{lemma}
	\small
	\begin{example}
		\begin{center}
		\begin{forest}
			parse tree
			[$\SF (x \lor \neg x)$ [$\SF x$ [$\SF \neg x$ [$\ST x$]]]]
		\end{forest}
		\end{center}
	\end{example}
\end{frame}

% В последствие ще използваме табло метода, за да намерим модел, в който дадена формула е изпълнима. Ще добавим няколко атомарни формули и ще търсим отворен клон. Ако не намерим такъв - нямаме модел за формулата. Ако има - формулата няма лексикални противоречия и почваме да строим модел удоволетворяващ атомарните формули в този клон.

% Всеки приключен клон в талбото е множество от атомарни формули. Таблото е множество от такива множества. Можем да си мислим за външното множество като дизункция на вътрешните множества, а вътрешните множества са  конюнкция на атомарни формули. По този начин таблото е генеризация на дизюнктивна нормална форма. По този начин преобразуваме формулата в дизюнктивна нормална форма.

\section{Контактна логика}
\subsection{Синтаксис}

\begin{frame}{Контактна логика - синтаксис}
	\begin{itemize}
		\item \textit{Булеви променливи} (изброимо множество $\V$)
		\item \textit{Булеви константи}: 0 и 1
		\item \textit{Булеви операции}:
		\begin{itemize}
			\item $\sqcap$ Сечение
			\item $\sqcup$ Обединение
			\item $^*$ Допълнение
		\end{itemize}
		\item \textit{Булеви термове}
		\item \textit{Логически връзки}: $\neg$, $\land$, $\lor$, $\Rightarrow$, $\Leftrightarrow$
		\item \textit{Логически константи}: $\top$ и $\bot$
		\item \textit{Модални връзки}: $\leq$(part-of) and $C$(contact)
		\item \textit{Формули}
	\end{itemize}
\end{frame}

\begin{frame}{Контактна логика - термове}
	\textbf{\textit{Терм}}
	\begin{itemize}
		\item Булева променлива
		\item Булева константа
		\item Ако $a$ е терм, то $a^*$ също е терм
		\item Ако $a$ и $b$ са термове, то и $a \sqcap b$ и $a \sqcup b$ са също термове
	\end{itemize}
\end{frame}

\begin{frame}{Контактна логика - формули}
	\textbf{\textit{Атомарни формули}} са от вида $a \leq b$ and $aCb$, където $a$ и $b$ са термове.

	\textbf{\textit{Формула}}
	\begin{itemize}
		\item Логическа константа
		\item Атомарна формула
		\item Ако $\phi$ е формула, то $\neg\phi$  съшо е формула
		\item Ако $\phi$ и $\psi$ са формули, то ($\phi \land \psi$), ($\phi \lor \psi$), ($\phi \Rightarrow \psi$) and ($\phi \Leftrightarrow \psi$) са също формули
	\end{itemize}
\end{frame}

\subsection{Семантика}

\subsection{Изпълнимост на формула}
\subsection{Алгоритъм за строене на модел}

\section{Свързана контактна логика}
\subsection{Свързаност}
\subsection{Алгоритъм за строене на свързан модел}

\section{Имплементация}
\subsection{Строене на формула от текст}
\subsection{Строене на свързан модел}
\subsection{Сървър и уеб приложение}
\subsection{Автоматична компилация, тестване и среда}

\end{document}