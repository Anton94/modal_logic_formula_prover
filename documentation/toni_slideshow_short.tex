\documentclass[14pt, aspectratio=169]{beamer}

% sudo apt-get install texlive-lang-cyrillic
\usepackage[T2A,T1]{fontenc}
\usepackage[utf8]{inputenc}
\usepackage[main=bulgarian,english]{babel}
\usepackage{algorithm,algorithmic}
\usepackage{hyperref}

\usepackage{forest}
\usepackage{mathtools}
\forestset{
    parse tree/.style={for tree={s sep=0.5em, minimum size=0.5em}}
}

\renewcommand\thealgorithm{}
\newcommand{\setalglineno}[1]{\setcounter{ALC@line}{\numexpr#1-1}}

\graphicspath{ {./image/} }

\usetheme{Boadilla}
\usecolortheme{beaver}

% Hide the navigation buttons at the bottom of each slide
\setbeamertemplate{navigation symbols}{}

% Lets the boolet points for next slides to be seen
%\setbeamercovered{transparent}

% Increase the size of title in the footer
\makeatletter
\defbeamertemplate*{footline}{Dan P theme}
{
  \leavevmode%
  \hbox{%
  \begin{beamercolorbox}[wd=.2\paperwidth,ht=2.25ex,dp=1ex,center]{author in head/foot}%
    \usebeamerfont{author in head/foot}\insertshortauthor\expandafter\beamer@ifempty\expandafter{\beamer@shortinstitute}{}{~~(\insertshortinstitute)}
  \end{beamercolorbox}%
  \begin{beamercolorbox}[wd=.6\paperwidth,ht=2.25ex,dp=1ex,center]{title in head/foot}%
    \usebeamerfont{title in head/foot}\insertshorttitle
  \end{beamercolorbox}%
  \begin{beamercolorbox}[wd=.2\paperwidth,ht=2.25ex,dp=1ex,right]{date in head/foot}%
    \usebeamerfont{date in head/foot}\insertshortdate{}\hspace*{2em}
\insertframenumber{} / \inserttotalframenumber\hspace*{2ex}
  \end{beamercolorbox}}%
  \vskip0pt%
}
\makeatother

% Teach babel the tags to translate them
\usepackage{etoolbox}
\newtheorem{axiom}{Axiom}
\newtheorem{step}[theorem]{Step}
\patchcmd{\theorem}{Theorem}{Теорема}{}{}
\patchcmd{\corollary}{Corollary}{Следствие}{}{}
\patchcmd{\lemma}{Lemma}{Лема}{}{}
\patchcmd{\proposition}{Proposition}{Предложение}{}{}
\patchcmd{\axiom}{Axiom}{Аксиома}{}{}
\patchcmd{\example}{Example}{Пример}{}{}
\patchcmd{\definition}{Definition}{Дефиниция}{}{}
\patchcmd{\remark}{Remark}{Забележка}{}{}
\patchcmd{\step}{Step}{Стъпка}{}{}

\newcommand\ST{\mathbb{T}}
\newcommand\SF{\mathbb{F}}
\newcommand\SB{\mathbb{B}}
\newcommand\M{\mathcal{M}}
\newcommand\F{\mathcal{F}}
\newcommand\p{\mathcal{P}}
\newcommand\V{\mathcal{V}}
\newcommand\E{\mathcal{E}}
\newcommand\Tb{\mathbf{T}}
\newcommand\Vb{\mathbf{V}}
\newcommand{\BranchConjunction}{%
\begin{equation*}
			\beta =
			\bigwedge_{\ST C(a,b)\in{\SB}} C(a, b) \:\: \wedge \:\:
			\bigwedge_{\ST d = 0\in{\SB}} d = 0 \:\: \wedge \:\:
			\bigwedge_{\SF C(e,f)\in{\SB}} \neg C(e, f) \:\: \wedge \:\:
			\bigwedge_{\SF g = 0\in{\SB}} g \neq 0
\end{equation*}%
}
\newcommand{\pair}[2]{ \langle #1, #2 \rangle }
\newcommand{\pairXY}{ \pair{x}{y} }
\newcommand{\pairXX}{ \pair{x}{x} }
\newcommand{\pairYX}{ \pair{y}{x} }
\newcommand{\pairYY}{ \pair{y}{y} }
\def\iff{\leftrightarrow}

\logo{\includegraphics[width=1cm]{fmi_logo.png}}

\date[\today]{}

\title{\textsc{Уеб система за изпълнимост на контактната логика за свързаност}}

%\subject{}

%\institute{}

\author[Антон Дудов]{
	\textsc{Факултет по математика и информатика} \\
	\textsc{Катедра по математическа логика и приложенията ѝ} \\ [3mm]
	\textsc{Антон Дудов} \\ [3mm]
	\small
	\textsc{Магистърска програма Логика и алгоритми} \\
	\textsc{спец.~Информатика} \\
	\textsc{Факултетен номер: 25691} \\ [3mm]
	\small\textsc{Научен ръководител: проф. Тинко Тинчев}
}

\begin{document}

\begin{frame}
\titlepage
\end{frame}

% На кратко казано чрез табло методът свеждаме формулата в дизюнктивна нормална форма.
% Конюнктите, на която нямат атомарна формула и отрицание на същата атомарна формула.
% Ще разглеждаме конюнктите и ще се опитаме да създадем модел за някой от тях.

% Първо, трябва ни синтактичен способ, с който да покажем, че дадена формула не е изпълнима и да правим анализ по формулата.
% Ще следваме описанието на табло метода от книгата - Handbook of tableau methods за табло методите. Там биват два вида табла - със или без знаци. Ще ползваме версията със знаци.
% Знаците, с които ще маркираме формулите са F и T.
% Формулите със знаци са изрази от вида FX и TX

% Стандартният начин за визуализиране на табло метода е като двоично дърво, възлите на които са крайно множество от формули със знаци. Разклоненията са дизункциите във формулата, а клоните са конюнкция на формулите Ще разбиваме тези формули със знаци докато не останат само атомарни фомрули, т.е. само променливи със знаци.

% В последствие ще използваме табло метода, за да намерим модел, в който дадена формула е изпълнима. Ще добавим няколко атомарни формули и ще търсим отворен клон. Ако не намерим такъв - нямаме модел за формулата. Ако има - формулата няма лексикални противоречия и почваме да строим модел удовлетворяващ атомарните формули в този клон.

% Всеки приключен клон в талбото е множество от атомарни формули. Таблото е множество от такива множества. Можем да си мислим за външното множество като дизункция на вътрешните множества, а вътрешните множества са  конюнкция на атомарни формули. По този начин таблото е генеризация на дизюнктивна нормална форма. По този начин преобразуваме формулата в дизюнктивна нормална форма.

\begin{frame}{Табло метод съждителна логика}
	\begin{center}
	\begin{forest}
		parse tree
		[$\ST ((x \land \neg x) \lor (\neg x \land y))$
			[$\ST (x \land \neg x)$ [$\ST x \mbox{,} \ST \neg x$ [$\ST x \mbox{,} \SF x$]]]
			[$\ST (\neg x \land y)$ [$\ST \neg  x \mbox{,} \ST y$ [$\SF x \mbox{,} \ST y$]]]
		]
	\end{forest}
	\end{center}
\end{frame}


% ВАЖНО - два типа обекти, булеви термове за означаване на региони и формули за означаване на свойства на региони . Атомарните формули се получават като приложим модална връзка към двойка термове
%\begin{frame}{Контактна логика - синтаксис}
%	\begin{itemize}
%		\item \textit{Булеви променливи} (изброимо множество $\V$)
%		\item \textit{Булеви константи}: 0 и 1
%		\item \textit{Булеви операции}:
%		\begin{itemize}
%			\item $\sqcap$ Сечение
%			\item $\sqcup$ Обединение
%			\item $^*$ Допълнение
%		\end{itemize}
%		\item \textit{Булеви термове}
%		\item \textit{Логически връзки}: $\neg$, $\land$, $\lor$, $\Rightarrow$, $\Leftrightarrow$
%		\item \textit{Логически константи}: $\top$ и $\bot$
%		\item \textit{Модални връзки}: $\leq$ и $C$ (атомарни формули)
%		\item \textit{Формули}
%	\end{itemize}
%\end{frame}

% ВАЖНО контактните логики имат топологична, алгебрачни и релационна семантика . ние ще разглеждаме релационната семантика
% R е рефлексивна и симетрична релация
% теоретико множествената разлика е W \ v(a) или от W изхвърляме точките от v(a)

%\begin{frame}{Контактна логика - релационна семантика}
%	\uncover<1->{
%		$\F$ = (W, R) е релационна система с W $\neq$ $\emptyset$ и R $\subseteq$ $W^2$, реф. и сим.
%	}
%
%	\uncover<2->{
%		\begin{definition}[Оценка]
%			\textbf{\textit{Оценка}} на булеви променливи в $\F$ е всяка функция $\upsilon : \V \rightarrow \p(W)$. Разширяваме $\upsilon$ индуктивно за булевите термове:
%
%			\begin{itemize}
%				\item $\upsilon(0) = \emptyset$
%				\item $\upsilon(1) = W$
%				\item $\upsilon(a \sqcap b) = \upsilon(a) \cap \upsilon(b)$
%				\item $\upsilon(a \sqcup b) = \upsilon(a) \cup \upsilon(b)$
%				\item $\upsilon(a^*) = W \setminus \upsilon(a)$
%			\end{itemize}
%		\end{definition}
%	}
%\end{frame}

%\begin{frame}{Контактна логика - модел}
%	\begin{definition}[Модел]
%		$\M$ = ($\F, \upsilon$) се нарича \textbf{\textit{модел}}.
%
%		Истиността на формула $\phi$ в $\M$ ($\M \models \phi$) се дефинира индуктивно за всички формули както следва:
%		\begin{itemize}
%			\item $\M \models \top$
%			\item $\M \not\models \bot$
%			\item $\M \models a \leq b \iff \upsilon(a) \subseteq \upsilon(b)$
%			\item $\M \models aCb \iff (\exists x \in \upsilon(a))(\exists y \in \upsilon(b))(xRy)$
%		\end{itemize}
%	\end{definition}
%\end{frame}
%
%\begin{frame}{Контактна логика - модел}
%	\begin{definition}[Модел]
%		\begin{itemize}
%			\item $\M \models \neg \phi \iff \M \not\models \phi$
%			\item $\M \models \phi \land \psi \iff \M \models \phi$ и $\M \models \psi$
%			\item $\M \models \phi \lor \psi \iff \M \models \phi$ или $\M \models \psi$
%			\item $\M \models \phi \Rightarrow \psi \iff \M \not\models \phi$ или $\M \models \psi$
%			\item $\M \models \phi \Leftrightarrow \psi \iff$ ($\M \models \phi$ и $\M \models \psi$) или ($\M \not\models \phi$ и $\M \not\models \psi$)
%		\end{itemize}
%	\end{definition}
%\end{frame}

% В модел фи е вярна (изводима е формално, а сме доста далеч от формалността)

%\begin{frame}{Контактна логика - изпълнимост на формула}
%	\begin{definition}[Модел на формула]
%		Модел $\M$ е \textit{\textbf{модел на формулата}} $\phi$, ако $\phi$ е \textit{вярвна} в~$\M$.
%	\end{definition}
%	\begin{definition}[Изпълнимост на формула]
%		Ако $\phi$ има модел $\M$, то $\phi$ е \textbf{изпълнима}.
%	\end{definition}
%\end{frame}

%\begin{frame}{Контактна логика - нулев терм}
%	Нека a и b са термове. Ясно е, че:
%	\begin{itemize}
%		\item $\M \models a = b \iff \upsilon(a) = \upsilon(b)$
%		\item $\M \models a \leq b \iff \M \models a \sqcap b^*=0$ (нулев терм)
%		\item $\M \models \neg(a \leq b) \iff \M \models  a \sqcap b^* \neq 0$ (ненулев терм)
%	\end{itemize}
%\end{frame}
%
%\begin{frame}{Контактна логика - свойства на релацията}
%	Нека a и b са термове.
%	\begin{itemize}
%		\item $\M \models a \neq 0 \iff \M \models aCa$
%		\item $\M \models aCb \iff \M \models bCa$
%	\end{itemize}
%	\bigskip
%	Следват от рефлексивността и симетричността на R.
%\end{frame}
%
%\begin{frame}{Контактната логика за свързаност}
%	\begin{lemma}
%		Релационната система е свързана точно тогава, когато за всяка оценка в нея следната формула е вярна
%		\begin{equation*}
%			a \neq 0 \land a \neq 1 \rightarrow aCa^*
%		\end{equation*}
%
%		На тази формула ще ѝ казваме аксиома за свързаност.
%	\end{lemma}
%
%	\begin{equation*}
%		\upsilon(a) \neq \emptyset \land \upsilon(a) \neq W \rightarrow (\exists x \in \upsilon(a)) (\exists y \in W \setminus \upsilon(a))(xRy)
%	\end{equation*}
%\end{frame}
%
%\begin{frame}
%	\uncover<1->{
%		Релационната система $\F = (W, R)$ дефинира неориентиран граф G(W, R). $W$ е множеството от върхове, а $R$ е множеството от ребра.
%	}
%
%	\uncover<2->{
%		\begin{definition}[Път в граф]
%			Нека G = (W, R) е граф. \textbf{Път} $\pi_G(x, y)$ е поредица от върхове $(x, v_1, \dotsc , v_k, y)$, такива че $x,  v_1,\dotsc,v_k, y \in V$ и $xRv1,v_1Rv_2,\dotsc, v_{k-1}Rv_k, v_kRy$.
%		\end{definition}
%	}
%
%	\uncover<3->{
%		\begin{definition}[Свързан граф]
%			Нека $G = (W, R)$ е неориентиран граф. $G$ е \textbf{свързан}, ако има път между всеки два различни върха в $W$.
%			\begin{equation*}
%				x, y \in W \rightarrow (x \neq y \implies \pi_G(x, y))
%			\end{equation*}
%		\end{definition}
%	}
%\end{frame}
%
%% Тук казвам, че че модалните точки са 2^n и всички подмножества от модални точки са 2^2^n и е твърде много. По-разумно ще е да направим модел с всички валидни модални точки и релации между тях и да търсим в него свързан подмодел
%
%\begin{frame}{Свързаност}
%	\uncover<1->{
%		\begin{theorem}[Свързаност]
%		 Нека $\F = (W, R)$ е релационна система и G = (W, R) е графът, дефиниран от нея.
%			\begin{center}
%				аксиомата за свързаност е удовлетворена в $\F \Leftrightarrow $ G е свързан
%			\end{center}
%		\end{theorem}
%	}
%
%	\uncover<2->{
%		\begin{definition}[Свързан модел]
%			Нека $\F$ = (W, R) е релационна система. Нека G = (W, R) е графът дефиниран от нея $\F$. Нека $\M = (\F, \upsilon)$ е модел на $\beta$. $\M$ е \textbf{свързан модел} на $\beta$, ако G e свързан граф.
%		\end{definition}
%	}
%\end{frame}
%
\begin{frame}{Строене на модел}
	\begin{center}
		Формула $\phi \rightarrow$ табло с начало $\ST \phi \rightarrow$ отворен клон $\SB$.
	\end{center}

	\begin{itemize}
		\item $\ST C(a,b)$ $\rightarrow$ $C(a,b)$ (контакт)
		\item $\SF C(e,f)$ $\rightarrow$ $\neg C(e,f)$ (не-контакт)
		\item $\ST a \leq b$ $\rightarrow$ $a \leq b$ $\rightarrow$ $a \sqcap b^*=0$ $\rightarrow$ $g = 0$ (нулев терм)
		\item $\SF a \leq b$ $\rightarrow$ $\neg (a \leq b)$ $\rightarrow$ $a \sqcap b^* \neq 0$ $\rightarrow$ $d \neq 0$ (ненулев терм)
	\end{itemize}

	\BranchConjunction
\end{frame}

% ТУК трябва да кажа идеята на алгоритъма за строене на модел.

% Да си мислим за формула с N променливи. За верността на формулата имат значение само оценките на тези променливи. В този смисъл една модална точка казва кои променливи са верни в нея и кои не са верни. Точките може да си мислим като характеристичния вектор на верността на променливите в тази точка

\begin{frame}{Модални точки}
	\begin{definition}[Модална точка]
		Оценка на променливи $\E_n$ за $n$ булеви променливи е поредица от единици и нули както следва:
		\begin{equation*}
			\E_n = \; < e_1, e_2, \ldots , e_n >, \; where \; e_1, \ldots, e_n \in \{0, 1 \}
		\end{equation*}
		\textbf{Модална точка} е оценка на променливи $\E_n$
	\end{definition}
\end{frame}

%\begin{frame}{Модални точки}
%	\uncover<1->{
%		\begin{definition}[$W_n$]
%			Множеството от всички модални точки за n променливи е \textbf{$W_n$}
%			\begin{equation*}
%				W_n = \{< e_1, e_2, \ldots , e_n > \mid e_1, \ldots, e_n \in \{0, 1 \} \}
%			\end{equation*}
%			$|W_n| = 2^n$
%		\end{definition}
%	}
%
%	\uncover<2->{
%		\begin{definition}
%			$(\E_n)^i$ e i-тия елемент в поредицата $\E_n$.
%		\end{definition}
%	}
%\end{frame}

\begin{frame}{Оценка}
	\textbf{Оценка $\upsilon : \V \rightarrow \p(W)$}:
	\begin{center}
		$\upsilon(x_i) = \{ \E \mid \E \in W$ и $(\E)^i = 1 \}, \;\; x_i \in \V$
	\end{center}
	Дефинира се индуктивно за термове както следва:
	\begin{itemize}
		\item $\upsilon(0) = \emptyset$
		\item $\upsilon(1) = W$
		\item $\upsilon(a \sqcap b) = \upsilon(a) \cap \upsilon(b)$
		\item $\upsilon(a \sqcup b) = \upsilon(a) \cup \upsilon(b)$
		\item $\upsilon(a^*) = W \setminus \upsilon(a)$
	\end{itemize}
\end{frame}

%\begin{frame}{Изпълнимост на атомарни формули}
%	\begin{lemma}[Изпълнимост на нулевите термове]
%		\begin{equation*}
%			d = 0 \in \beta \rightarrow \upsilon(d) = \emptyset
%		\end{equation*}
%	\end{lemma}
%	\begin{lemma}[Изпълнимост на ненулевите термове]
%		\begin{equation*}
%			g \neq 0 \in \beta \rightarrow \upsilon(g) \neq \emptyset
%		\end{equation*}
%	\end{lemma}
%\end{frame}

\begin{frame}{Валидна модална точка}
	\begin{definition}[Валидна модална точка]
		$\E \in W_n$ е \textbf{валидна модална точка} на $\beta$, ако запазва изпълнимостта на нулевите термове и не-контактите
		\begin{equation*}
			g = 0 \in \beta \rightarrow \E \notin \upsilon(g) И
		\end{equation*}
		\begin{equation*}
			\neg C(e, f) \in \beta \rightarrow \E \notin (\upsilon(e) \cap \upsilon(f))
		\end{equation*}
	\end{definition}

	\begin{definition}[$W^v$]
		Множеството от всички валидни модални точки е \textbf{$W^v$}
		\begin{center}
			$W^v = \{ \E \mid \E \in W_n$ и $\E$ е валидна модална точка на $\beta \}$
		\end{center}
	\end{definition}
\end{frame}

\begin{frame}{Валидна релация между точки}
	\begin{definition}[Валидна релация]
		Нека х, y $\in W^v$. Тогава $\pairXY$ е \textbf{валидна релация} на $\beta$, ако запазва изпълнимостта на не-контактите в $\beta$.
		\begin{center}
			$\neg C(e, f) \in \beta \rightarrow \neg((x \in \upsilon(e)$ и $y \in \upsilon(f))$ или $(x \in \upsilon(f)$ и $y \in \upsilon(e)))$
		\end{center}
	\end{definition}

	\begin{definition}[$R^v$]
		\begin{center}
			$R^v = \{ \pairXY  \mid x, y \in W^v$ и $\pairXY$  е валидна релация на $\beta\}$
		\end{center}
	\end{definition}
\end{frame}

\begin{frame}{Свързан модел}
	\small
	\begin{step}
		$\F^v = (W^v, R^v)$, $\M^v = (\F^v, \upsilon)$. $\M^v$ е модел на $\beta$, ако контактите и ненулевите термове в $\beta$ са удовлетворени. Ако $\M^v$ не е модел, тогава $\beta$ няма модел(нито свързан модел).
	\end{step}
%\end{frame}

%\begin{frame}{Подграф}
%	\begin{definition}[Подграф]
%		$G'(W', R') \subseteq G(W, R)$, ако:
%		\begin{center}
%			$W' \subseteq W$ и $R' = \{ \pairXY  \mid x, y \in W'$ и $xRy \}$
%		\end{center}
%	\end{definition}
%\end{frame}

%\begin{frame}{Подмодел}
%	\begin{lemma}[Подмодел]
%		$\F = (W, R)$, $\M = (\F, \upsilon)$. Нека $G'=(W',R') \subseteq G=(W, R)$. Тогава G' дефинира модел $\M' = ((W', R'), \upsilon')$, където:
%		\begin{itemize}
%			\item $\upsilon'(x) = \upsilon(x) \cap W'$, за всяка променлива х
%			\item $\upsilon'(0) = \emptyset$
%			\item $\upsilon'(1) = W'$
%			\item $\upsilon'(a \sqcap b) = \upsilon'(a) \cap \upsilon'(b)$
%			\item $\upsilon'(a \sqcup b) = \upsilon'(a) \cup \upsilon'(b)$
%			\item $\upsilon'(a^*) = W' \setminus \upsilon'(a)$
%		\end{itemize}
%	\end{lemma}
%\end{frame}
%
%\begin{frame}
%	\begin{lemma}[Запазване удовлетворимостта на атомарните формули]
%		$G^v=(W^v, R^v)$ е графът породен от $\F^v$. Нека $G=(W,R) \subseteq G^v$ и $\M=((W,R), \upsilon')$ е моделът дефиниран от G. Тогава:
%		\begin{itemize}
%			\item<1-> M запазва удовлетворимостта на контактите в $\beta$, ако
%			\begin{equation*}
%				C(a,b) \in \beta \rightarrow (\exists x \in \upsilon'(a))(\exists y \in \upsilon'(b))(xRy)
%			\end{equation*}
%
%			\item<2-> M запазва удовлетворимостта на ненулевите термове в $\beta$, ако
%			\begin{equation*}
%				g \neq 0 \in \beta \rightarrow \upsilon'(g) \neq \emptyset
%			\end{equation*}
%		\end{itemize}
%	\end{lemma}
%\end{frame}
%
%\begin{frame}{Свързани компоненти - дефиниции}
%	\uncover<1->{
%		\begin{definition}[Свързана компонента]
%			Нека $G=(W, R)$ е граф. Нека $G'=(W', R') \subseteq G(W, R)$. Ако $G'$ е свързан, то G' е \textbf{свързана компонента} на G.
%		\end{definition}
%	}
%
%	\uncover<2->{
%		\begin{definition}[Максимална свързана компонента]
%			Нека $G=(W, R)$ е граф. Нека $G'=(W', R')$ е свързана компонента на G. $G'$ е \textbf{максимална свързана компонента} на G, ако:
%			\begin{gather*}
%				x \in W' \rightarrow \neg(\exists y \in W \setminus W')(xRy) \mbox{ и} \\ 
%				x,y \in W' \rightarrow xRy \iff xR'y
%			\end{gather*}
%		\end{definition}
%	}
%\end{frame}

%\begin{frame}{Свързан модел}
	\begin{step}
		Нека $\M^v$ е модел на $\beta$. Всички модели, дефинирани от свързаните компоненти на $G^v$, запазват удовлетворимостта на нулевите термове и не-контактите(не добавят точки, нито релации). Ако има свързана компонента, която запазва удовлетворимостта на контактите и ненулевите термове, то тя дефинира \textbf{свързан модел} на $\beta$. Достатъчно е да разгледаме само максималните свързани компоненти на $G^v$.
	\end{step}
\end{frame}

\begin{frame}{Имплементация}
	\begin{itemize}
		\item Flex \& Bison за строене на AST (Абстрактно синтактично дърво)
		\item Превръщане на AST формула във формула с удобни и ефективни операции свързани за табло метода и строенето на модела
		\item Пускане на табло метода за търсене на отворен клон
		\item Генериране на (свързан) модел
		\item Компилиране на библиотеката в WebAssembly
		\item Уеб приложение
		\item Тестове
		\item Автоматични билдове
		\item \url{https://github.com/Anton94/modal_logic_formula_prover}
	\end{itemize}
\end{frame}

\begin{frame}{Демо}
	\begin{center}
		Демо - \url{http://logic.fmi.uni-sofia.bg/theses/Dudov_Stoev/}
	\end{center}
\end{frame}

\begin{frame}
	\begin{center}
		\Huge
		Благодаря за вниманието!
		\\ [12mm]
		Въпроси?
	\end{center}

    \mbox{}
    \\ [20mm]
	\small
	Repository - \url{https://github.com/Anton94/modal_logic_formula_prover}
\end{frame}

\end{document}