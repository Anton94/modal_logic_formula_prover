\documentclass[14pt, aspectratio=169]{beamer}

% sudo apt-get install texlive-lang-cyrillic
\usepackage[T2A,T1]{fontenc}
\usepackage[utf8]{inputenc}
\usepackage[main=bulgarian,english]{babel}
\usepackage{algorithm,algorithmic}
\usepackage{hyperref}

\usepackage{forest}
\usepackage{mathtools}
\forestset{
    parse tree/.style={for tree={s sep=0.5em, minimum size=0.5em}}
}

\renewcommand\thealgorithm{}
\newcommand{\setalglineno}[1]{\setcounter{ALC@line}{\numexpr#1-1}}

\graphicspath{ {./image/} }

\usetheme{Boadilla}
\usecolortheme{beaver}

% Hide the navigation buttons at the bottom of each slide
\setbeamertemplate{navigation symbols}{}

% Lets the boolet points for next slides to be seen
%\setbeamercovered{transparent}

% Increase the size of title in the footer
\makeatletter
\defbeamertemplate*{footline}{Dan P theme}
{
  \leavevmode%
  \hbox{%
  \begin{beamercolorbox}[wd=.2\paperwidth,ht=2.25ex,dp=1ex,center]{author in head/foot}%
    \usebeamerfont{author in head/foot}\insertshortauthor\expandafter\beamer@ifempty\expandafter{\beamer@shortinstitute}{}{~~(\insertshortinstitute)}
  \end{beamercolorbox}%
  \begin{beamercolorbox}[wd=.6\paperwidth,ht=2.25ex,dp=1ex,center]{title in head/foot}%
    \usebeamerfont{title in head/foot}\insertshorttitle
  \end{beamercolorbox}%
  \begin{beamercolorbox}[wd=.2\paperwidth,ht=2.25ex,dp=1ex,right]{date in head/foot}%
    \usebeamerfont{date in head/foot}\insertshortdate{}\hspace*{2em}
\insertframenumber{} / \inserttotalframenumber\hspace*{2ex}
  \end{beamercolorbox}}%
  \vskip0pt%
}
\makeatother

% Teach babel the tags to translate them
\usepackage{etoolbox}
\newtheorem{axiom}{Axiom}
\newtheorem{step}[theorem]{Step}
\patchcmd{\theorem}{Theorem}{Теорема}{}{}
\patchcmd{\corollary}{Corollary}{Следствие}{}{}
\patchcmd{\lemma}{Lemma}{Лема}{}{}
\patchcmd{\proposition}{Proposition}{Предложение}{}{}
\patchcmd{\axiom}{Axiom}{Аксиома}{}{}
\patchcmd{\example}{Example}{Пример}{}{}
\patchcmd{\definition}{Definition}{Дефиниция}{}{}
\patchcmd{\remark}{Remark}{Забележка}{}{}
\patchcmd{\step}{Step}{Стъпка}{}{}

\newcommand\ST{\mathbb{T}}
\newcommand\SF{\mathbb{F}}
\newcommand\SB{\mathbb{B}}
\newcommand\M{\mathcal{M}}
\newcommand\F{\mathcal{F}}
\newcommand\p{\mathcal{P}}
\newcommand\V{\mathcal{V}}
\newcommand\E{\mathcal{E}}
\newcommand\Tb{\mathbf{T}}
\newcommand\Vb{\mathbf{V}}
\newcommand{\BranchConjunction}{%
\begin{equation*}
			\beta =
			\bigwedge_{\ST C(a,b)\in{\SB}} C(a, b) \:\: \wedge \:\:
			\bigwedge_{\ST d = 0\in{\SB}} d = 0 \:\: \wedge \:\:
			\bigwedge_{\SF C(e,f)\in{\SB}} \neg C(e, f) \:\: \wedge \:\:
			\bigwedge_{\SF g = 0\in{\SB}} g \neq 0
\end{equation*}%
}
\newcommand{\pair}[2]{ \langle #1, #2 \rangle }
\newcommand{\pairXY}{ \pair{x}{y} }
\newcommand{\pairXX}{ \pair{x}{x} }
\newcommand{\pairYX}{ \pair{y}{x} }
\newcommand{\pairYY}{ \pair{y}{y} }
\def\iff{\leftrightarrow}

\logo{\includegraphics[width=1cm]{fmi_logo.png}}

\date[\today]{}

\title{\textsc{Уеб система за изпълнимост в контактната логика на свързаността}}

%\subject{}

%\institute{}

\author[Антон Дудов]{
	\textsc{Факултет по математика и информатика} \\
	\textsc{Катедра по математическа логика и приложенията ѝ} \\ [3mm]
	\textsc{Антон Дудов} \\ [3mm]
	\small
	\textsc{Магистърска програма Логика и алгоритми} \\
	\textsc{спец.~Информатика} \\
	\textsc{Факултетен номер: 25691} \\ [3mm]
	\small\textsc{Научен ръководител: проф. Тинко Тинчев}
}

\begin{document}

\begin{frame}
\titlepage
\end{frame}

% Няколко реда какво очаква човек в тази дипломна работа(в документацията).

% Целта на дипломната работа е един алгоритъм за проверка на изпълнимост за формула в контактната логика  на свързаността, да бъде програмно реализиран. За целта е реализиран табло метод за съждителната логика и е разширен първо за контактна логика и второ за контактната логика на свързаността. 
% ще видите в демонстрацията графично представяне на модела и по -подробна информация.

%Ние ще използваме табло с формули маркирани( или нарочени) за верни съотвено неверни . Табло метод от този вид записваме формулата , норочваме я за вярно и прилагаме правила. В резултат, на което поолучаваме двоично дърво, във върховете на дървото има такива и такива неща.

% Да добавя още няколко примера да има за всеки случай.

% Стандартният начин за визуализиране на табло метода е като двоично дърво, възлите на които са крайно множество от формули със знаци. За контактната логика възлите се разширяват с атомарните формули <= и C.

\begin{frame}{Табло метод съждителна логика}
	\begin{center}
	\begin{forest}
		parse tree
		[$\ST ((x \land \neg x) \lor (\neg x \land y))$
			[$\ST (x \land \neg x)$ [$\ST x \mbox{,} \ST \neg x$ [$\ST x \mbox{,} \SF x$]]]
			[$\ST (\neg x \land y)$ [$\ST \neg  x \mbox{,} \ST y$ [$\SF x \mbox{,} \ST y$]]]
		]
	\end{forest}
	\end{center}
\end{frame}

\begin{frame}{Табло метод контактна логика - листо}
	\begin{center}
		Формула $\phi \rightarrow$ табло с начало $\ST \phi \rightarrow$ листо $\SB$ в отворен клон.
	\end{center}

	\begin{itemize}
		\item $\ST C(a,b)$ $\rightarrow$ $C(a,b)$ (контакт)
		\item $\SF C(e,f)$ $\rightarrow$ $\neg C(e,f)$ (не-контакт)
		\item $\ST a \leq b$ $\rightarrow$ $a \leq b$ $\rightarrow$ $a \sqcap b^*=0$ $\rightarrow$ $g = 0$ (нулев терм)
		\item $\SF a \leq b$ $\rightarrow$ $\neg (a \leq b)$ $\rightarrow$ $a \sqcap b^* \neq 0$ $\rightarrow$ $d \neq 0$ (ненулев терм)
	\end{itemize}

	\BranchConjunction
\end{frame}

% Да си мислим за формула с N променливи. За верността на формулата имат значение само оценките на тези променливи. В този смисъл една модална точка казва кои променливи са верни в нея и кои не са верни. Точките може да си мислим като характеристичния вектор на верността на променливите в тази точка.

\begin{frame}{Модални точки}
	\begin{definition}[Модална точка]
		Оценка на променливи $\E_n$ за $n$ булеви променливи е поредица от единици и нули както следва:
		\begin{equation*}
			\E_n = \; < e_1, e_2, \ldots , e_n >, \; where \; e_1, \ldots, e_n \in \{0, 1 \}
		\end{equation*}
		\textbf{Модална точка} е оценка на променливи $\E_n$
	\end{definition}
\end{frame}

\begin{frame}{Оценка}
	\textbf{Оценка $\upsilon : \V \rightarrow \p(W)$}:
	\begin{center}
		$\upsilon(x_i) = \{ \E \mid \E \in W$ и $(\E)^i = 1 \}, \;\; x_i \in \V$
	\end{center}
	Дефинира се индуктивно за термове както следва:
	\begin{itemize}
		\item $\upsilon(0) = \emptyset$
		\item $\upsilon(1) = W$
		\item $\upsilon(a \sqcap b) = \upsilon(a) \cap \upsilon(b)$
		\item $\upsilon(a \sqcup b) = \upsilon(a) \cup \upsilon(b)$
		\item $\upsilon(a^*) = W \setminus \upsilon(a)$
	\end{itemize}
\end{frame}

\begin{frame}{Валидна модална точка}
	\begin{definition}[Валидна модална точка]
		$\E \in W_n$ е \textbf{валидна модална точка} на $\beta$, ако запазва изпълнимостта на нулевите термове и не-контактите
		\begin{equation*}
			g = 0 \in \beta \rightarrow \E \notin \upsilon(g) И
		\end{equation*}
		\begin{equation*}
			\neg C(e, f) \in \beta \rightarrow \E \notin (\upsilon(e) \cap \upsilon(f))
		\end{equation*}
	\end{definition}

	\begin{definition}[$W^v$]
		Множеството от всички валидни модални точки е \textbf{$W^v$}.
	\end{definition}
\end{frame}

\begin{frame}{Валидна релация между точки}
	\begin{definition}[Валидна релация]
		Нека х, y $\in W^v$. Тогава $\pairXY$ е \textbf{валидна релация} на $\beta$, ако запазва изпълнимостта на не-контактите в $\beta$.
		\begin{center}
			$\neg C(e, f) \in \beta \rightarrow \neg((x \in \upsilon(e)$ и $y \in \upsilon(f))$ или $(x \in \upsilon(f)$ и $y \in \upsilon(e)))$
		\end{center}
	\end{definition}

	\begin{definition}[$R^v$]
		\begin{center}
			$R^v = \{ \pairXY  \mid x, y \in W^v$ и $\pairXY$  е валидна релация на $\beta\}$
		\end{center}
	\end{definition}
\end{frame}

\begin{frame}{Свързан модел}
	\small
	\begin{step}
		$\F^v = (W^v, R^v)$, $\M^v = (\F^v, \upsilon)$. $\M^v$ е модел на $\beta$, ако контактите и ненулевите термове в $\beta$ са удовлетворени. Ако $\M^v$ не е модел, тогава $\beta$ няма модел(нито свързан модел).
	\end{step}

	\begin{step}
		Нека $\M^v$ е модел на $\beta$. Всички модели, дефинирани от свързаните компоненти на $G^v$, запазват удовлетворимостта на нулевите термове и не-контактите(не добавят точки, нито релации). Ако има свързана компонента, която запазва удовлетворимостта на контактите и ненулевите термове, то тя дефинира \textbf{свързан модел} на $\beta$. Достатъчно е да разгледаме само максималните свързани компоненти на $G^v$.
	\end{step}
\end{frame}

\begin{frame}{Имплементация}
	\begin{itemize}
		\item Flex \& Bison за строене на AST (Абстрактно синтактично дърво)
		\item Превръщане на AST формула във формула с удобни и ефективни операции свързани за табло метода и строенето на модела
		\item Пускане на табло метода за търсене на отворен клон
		\item Генериране на (свързан) модел
		\item Компилиране на библиотеката в WebAssembly
		\item Уеб приложение
		\item Тестове
		\item Автоматични билдове
		\item \url{https://github.com/Anton94/modal_logic_formula_prover}
	\end{itemize}
\end{frame}

\begin{frame}{Демо}
	\begin{center}
		Демо - \url{http://logic.fmi.uni-sofia.bg/theses/Dudov_Stoev/}
	\end{center}
\end{frame}

\begin{frame}
	\begin{center}
		\Huge
		Благодаря за вниманието!
		\\ [12mm]
		Въпроси?
	\end{center}

    \mbox{}
    \\ [20mm]
	\small
	Repository - \url{https://github.com/Anton94/modal_logic_formula_prover}
\end{frame}

\end{document}