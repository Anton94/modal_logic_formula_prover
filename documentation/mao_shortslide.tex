% This text is proprietary.
% It's a part of presentation made by myself.
% It may not used commercial.
% The noncommercial use such as private and study is free
% Dec 2007
% Author: Sascha Frank 
% University Freiburg 
% www.informatik.uni-freiburg.de/~frank/
%
% 
\documentclass{beamer}
\usepackage[utf8]{inputenc}
\usepackage[T2A,T1]{fontenc}
\usepackage[english, bulgarian]{babel}
\setbeamertemplate{navigation symbols}{}

\addtobeamertemplate{navigation symbols}{}{%
    \usebeamerfont{footline}%
    \usebeamercolor[fg]{footline}%
    \hspace{1em}%
    \insertframenumber/\inserttotalframenumber
}

\usepackage{mathtools}
\usepackage{amsmath}
\usepackage{amssymb}
\usepackage{amsthm}
\usepackage{systeme}
\usepackage{newtxtext}
\usepackage{newtxmath}
\usepackage{listings}
\usepackage{xcolor}
\usepackage{array}
\usepackage{makecell}
\usepackage{booktabs}% http://ctan.org/pkg/booktabs
\usepackage{float}
\usepackage{hyperref}
\usepackage{tikz}
\usepackage{multicol}
\usepackage{graphicx}

\usetikzlibrary{arrows.meta}
\graphicspath{ {./image/} }

\restylefloat{table}
\newcommand{\tabitem}{~~\llap{\textbullet}~~}

\newtheorem{defn}{Дефиниция}[section]
\newtheorem{axiom}{Аксиома}[section]
\newtheorem{prop}{Свойство}[section]
\newtheorem{lema}{Лема}[section]

\newcommand{\curvedE}{\mathscr{e}}
\newcommand{\vVar}{\mathscr{v}_{var}}
\newcommand{\Var}{\mathbb{V}ar}
\newcommand{\vE}{\mathscr{v}}
\newcommand{\vBool}{\xi}
\newcommand{\Ts}{\mathcal{T}_s}
\newcommand{\signT}{\mathbb{T}}
\newcommand{\signF}{\mathbb{F}}
\newcommand{\BranchConjunction}{%
\begin{align*}
			\bigwedge_{i=1}^{I} \; C(a_i, b_i) \:\: & \wedge \:\: 
			\bigwedge_{k=1}^{K} \; \neg C(e_k, f_k) \:\: \wedge \:\: \\
			\bigwedge_{j=1}^{J} \; d_j \neq 0 \:\: & \wedge \:\:
			\bigwedge_{l=1}^{L} \; g_l = 0 \:\:
\end{align*}%
}
\newcommand{\BranchConjunctionWithMeasure}{%
\begin{align*}
			\bigwedge_{i=1}^{I} \; C(a_i, b_i) \:\: & \wedge \:\: 
			\bigwedge_{k=1}^{K} \; \neg C(e_k, f_k) \:\: \wedge \:\: \\
			\bigwedge_{j=1}^{J} \; d_j \neq 0 \:\: & \wedge \:\:
			\bigwedge_{l=1}^{L} \; g_l = 0 \:\: \wedge \:\:\\
			\bigwedge_{p=1}^{P} \; m_p \le_\mu n_p \:\: &\wedge \:\:
			\bigwedge_{q=1}^{Q} \; u_q <_\mu v_q \:\:
\end{align*}%
}

\lstset { %
    literate={~} {$\sim$}{1},
    language=C++,
    backgroundcolor=\color{black!5}, % set backgroundcolor
    basicstyle=\footnotesize,% basic font setting
}



\usetheme{Montpellier}

\beamersetuncovermixins{\opaqueness<1>{25}}{\opaqueness<2->{15}}
\begin{document}
\title{Уеб система за изпълнимост в контактната логика с мярка}  
\author{\textsc{Факултет по математика и информатика} \\
	\textsc{Катедра по математическа логика и приложенията ѝ} \\ [3mm]
	\textsc{Стоев Мартин} \\ [3mm]
	\textsc{Магистерска програма Логика и Алгоритми, Информатика, Факултетен Номер: 25790} \\ [4mm]
	\small\textsc{Научен ръководител: проф. Тинко Тинчев}}
\date{\today} 
\begin{frame}
\titlepage
\end{frame}

%\begin{frame}\frametitle{Съдържание}\tableofcontents\end{frame} 

\section{Табло}
\subsection{Табло метод за съждителната логика}
\begin{frame}\frametitle{Табло метод за съждителната логика}
Маркиране на валидността на формула $\varphi$
\begin{itemize}
	\item $\signT \varphi$ - маркиране на формулата $\varphi$ за вярна
	\item $\signF \varphi$ - маркиране на формулата $\varphi$ за невярна
\end{itemize}
\vspace{20px}
Стъпки на табло метода
\begin{itemize}
	\item Правила
	\item Разширяване на дърво, което строи табло методът
	\item Намиране на противоречия
\end{itemize}
\end{frame}

\subsection{Табло метод за контактната логика}
\begin{frame}\frametitle{Табло метод за контактната логика}
Вид на табло клон:
\begin{itemize}
	\item $\signT C(a, b)$
	\item $\signF C(a, b)$ 
	\item $\signT а \leq b \iff \signT a \sqcap b* = 0$
	\item $\signF t = 0 \iff \signF a \sqcap b* = 0$
	\item $\signT a  \le_\mu b$
	\item $\signF a <_\mu b$
\end{itemize}
\end{frame}

\begin{frame}
	\begin{center}
	За дадена формула $\varphi$ търсим: $\mathcal{M} = (W, R, \mathscr{v})$ \\
	$ \mathcal{M} \models \varphi$
	\end{center}
\end{frame}

\begin{frame}\frametitle{Построяване на система линейни неравенства}
	Нека x и y са два терма, тогава формулата $\le_\mu(x, y)$ се преобразува в:
	\begin{align*}
		\sum_{i: p_i \in \vE(x)} X_i \leq \sum_{j: p_j \in \vE(y)} X_j
	\end{align*}

			Това преобразуване може да се опрости до:
			\begin{align*}
				\sum_{i: p_i \in \vE(x) \setminus \vE(y)} X_i \leq \sum_{j: p_j \in \vE(y) \setminus \vE(x)} X_j
			\end{align*}
\end{frame}

\begin{frame}
\begin{defn}[Множеството на всички променливи]
	С $\Var_B$ ще ознчаваме множеството от всички променливи използвани в табло клона B.
\end{defn}

\begin{defn}[Оценка на променливи]
	С p ще ознчаваме функцията, която за всяка променлива от $\Var_B$ дава истина или лъжа.
		\begin{align*}
			p : \Var_B \rightarrow \{ \textbf{лъжа}, \textbf{истина}\}
		\end{align*}
\end{defn}
\end{frame}

\begin{frame}
\begin{defn}[Булева оценка]
		Нека p е оценка на променливи и $\Ts$ е множеството от всички термове, тогава 
		функцията $\vBool_p : \Ts \rightarrow \{ \textbf{лъжа}, \textbf{истина}\}$ ще наричаме булева оценка, която се дефинира по следния начин:
		\begin{itemize}
			\item $\vBool_p(0) = \textbf{лъжа}$
			\item $\vBool_p(1) = \textbf{истина}$
			\item $\vBool_p(t) =p(t), \textit{ където } t \in \Var_B$
			\item $\vBool_p(a \sqcap b) = \vBool_p(a) \textit{ и } \vBool_p(b)$
			\item $\vBool_p(a \sqcup b) = \vBool_p(a) \textit{ или } \vBool_p(b)$
			\item $\vBool_p(a*) = \textit{не } \vBool_p(a)$
		\end{itemize}
\end{defn}
\end{frame}


\begin{frame}\frametitle{Изпълнимост в контактна логика с мярка}
	\begin{defn}%{Valid modal point}
	Нека B е клон в таблото. Казваме, че модалната точка p е валидна в B, когато:
	\begin{itemize}
		\item $t = 0 \in B:  \vBool_{p}(t) = \textbf{лъжа}$
		\item $\neg C(e, f) \in B: \vBool_{p}(e) = \textbf{лъжа} \textit{ или } \vBool_{p}(f) = \textbf{лъжа}$
	\end{itemize}
	\end{defn}

	\begin{defn}
		Нека B е клон в таблото и нека p и q са две валидни модални точки. Казваме, че $\langle p, q \rangle$ е валидна релация, когато:
		\begin{align*}
				\neg C(e, f) \in B: &(\vBool_{p}(e) = \textbf{лъжа} \textit{ или } \vBool_{q}(f) = \textbf{лъжа}) \textit{ и } \\
				 &(\vBool_{p}(f) = \textbf{лъжа} \textit{ или } \vBool_{q}(e) = \textbf{лъжа})
		\end{align*}
	\end{defn}
\end{frame}

\begin{frame}\frametitle{Изпълнимост в контактна логика с мярка}
	\begin{defn}
Нека B е клон в таблото и нека W е множество от валидни модални точки в B. Дефинираме модел $\mathcal{M} = (W, R, \mu, v)$ в B, където:
		\begin{align*}
				\vE(t) = \{ p \; | \; p \in W \textit{ и } &\vBool_{p}(t) =\; \textit{истина} \}, \\
				\textit{ където t е терм } & \textit{от атомарните формули в B}
		\end{align*}
		\begin{align*}
				R = \{ \langle p, q \rangle\; | \; p, q \in W \textit{ и } \langle p, q \rangle \textit{ е валидна релация}\}
		\end{align*}
	\end{defn}
\end{frame}

\begin{frame}
\begin{lema}[Невъзможни подмножествени модели]
Нека $\mathcal{M} = (W, R, \mu,\vE)$ е модел, където W е множество от валидни модални точки.
Нека $\mathcal{M'} = (W', R', \mu, \vE')$ е модел, където $W' \subseteq W, R' \subseteq R$, тогава:
	\begin{enumerate}
		\item $\mathcal{M} \not\models t \neq 0 \implies \mathcal{M'} \not\models t \neq 0$
		\item $\mathcal{M} \not\models C(a,b) \implies \mathcal{M'} \not\models C(a,b)$
	\end{enumerate}
	\end{lema}
	
	\begin{lema}[Дедукция на променливите]
		Нека $\mathcal{M} = (W, R, \mu, \vE)$ е модел, където W е множество от валидни точки.
		Нека $\mathcal{M'} = (W', R', \mu, \vE')$ е подмодел на $\mathcal{M}$:
	% NOTE kaji che R' e sechenie na R s W2
		\begin{align*}
			\vE'(t) =  \vE(t) \cap W'
		\end{align*}
	\end{lema}
\end{frame}

\begin{frame}\frametitle{Имплементация}
\begin{itemize}
	\item FLEX + BISON за построяване на формулата в AST (абстрактно синтактично дърво)
	\item Търсене на последователни атомарни клонове с табло метода
	\item Генериране на модел с мярка
	\item Kiwi библиотека за смятане на система линейни неравенства
	\item Уеб приложение за извикване на генерирането на модела и визуализиране на самия
\end{itemize}
 https://github.com/Anton94/modal\_logic\_formula\_prover
\end{frame}

\begin{frame}\frametitle{Демо}
\url{http://logic.fmi.uni-sofia.bg/theses/Dudov\_Stoev/}
\end{frame}

\begin{frame}
Благодаря за вниманието.
\end{frame}

\end{document}
























