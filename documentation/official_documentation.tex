\documentclass{article}
\usepackage{amssymb}
\usepackage{newtxtext}
\usepackage{newtxmath}

\title{Satisfiability Of Modal Logic Formulas}
\date{2019-9-28}
\author{Martin Stoev, Anton Dudov}

\begin{document}
	\maketitle
	\pagenumbering{gobble}
	\newpage

	\section{Formula Representation}
	Let $\mathbb{V}ar$ be the set of variables:
	\[ \mathbb{V}ar = \{p_0, p_1, p_2 ... \}\]
	Let $\mathbb{C}_t$ be the set of Term constants:
	\[ \mathbb{C}_t = \{0, 1 \}\]
	\subsection{Term recursive definition}
	\begin{itemize}
		\item $a \in \mathbb{C}_t$ is a term
		\item $p \in \mathbb{V}ar$ is a term
		\item If x is a term, then $\bar{x}$ is a term as well
		\item If x and y are terms, then $x \sigma y $ is a term as well,\\
			where $\sigma \: \in \: \{\sqcap, \sqcup\}$
	\end{itemize}
	Let $\mathbb{C}_f$ be the set of formula constants:
	\[ \mathbb{C}_f = \{T, F \}\]
	\subsection{Formula recursive definition}
	\begin{itemize}
		\item $a \in \mathbb{C}_f$ is a formula
		\item If x and y are terms, then C(x, y) is a formula
		\item If x and y are terms, then $x \le y$ is a formula
		\item If x and y are terms, then $x \le_m y$ is a formula
		\item If $\phi$ is a formula, then $\neg \phi$ is a formula as well
		\item If $\phi$ and $\psi$ are formulas, then $\phi \: \sigma \: \psi $ is a formula as well,\\
			where $\sigma \in \{\vee, \wedge, \rightarrow, \leftrightarrow\}$
	\end{itemize}
	
		\subsection{Definition: Atomic Formula}
			A formula will be called atomic formula if it is a constant or it is in one of the followings:
			\begin{itemize}
				\item C(x, y)
				\item $x \le y$
				\item $x \le_m y$
			\end{itemize}
			where x and y are terms.

	\section{Tableaux}
	The Tableaux process is decision procedure, which recursively breaks down a given formula into basic components 
	based on which a decision can be concluded. The recursive step which breaks down a formula creates one or two 
	new formulas, which in terms of their structure are simpler then the initial formula. Since the recursive step can create
	at most two new formulas, this means the recursive step will create at most two branches or a binary tree, where the nodes
	are the formulas and the links represent the recursive step. The different branches are considered to be disjuncted while
   	nodes of the same branch are considered in conjunction. The procedure modifies the tableau in such a way that the 
	formula represented by the resulting tableau is equivalent to the original one.

	Contradiction may arise when in the same branch, on some step there exists a formula and the negation of the same formula.
	If in some branch there exists a contradiction, then that branch closes. If all branches close then the proof is complete.
	
	The main principle of the tableaux is to break complex formulae into smaller ones until complementary pairs of literals are
	produced or no further expansion is possible.

	\subsection{Definition: Tableaux Step}
		The Tableaux Step takes as input a formula and a set of accumulated formulae and produces as 
		output one or two new formulae, depending on the operation. 
		The set of accumulated formulae consist of the broken down formulae by previous tableaux steps.
		The output of the tableaux step depends on the rule applied to the formula.

		\subsubsection{Rules}
		\subsubsection*{Negation}
			$\begin{array}{rl}
				& \neg \varphi, X \\
			      \cline{2-2}
			      & \varphi, X
			\end{array}$

		\subsubsection*{And}
			$\begin{array}{rl}
				& \varphi \: \wedge \: \psi, X \\
			      \cline{2-2}
			      & \varphi, \psi, X
			\end{array}$

		\subsubsection*{Or}
			$\begin{array}{rl}
				& \:\:\: \varphi \: \vee \: \psi, X \\
			      \cline{2-2}
			      & \varphi, X \:\:\:\:\:\:\:\: \psi, X
			\end{array}$

		\subsubsection*{Implication}
			$\begin{array}{rl}
				& \:\:\: \varphi \: \rightarrow \: \psi, X \\
			      \cline{2-2}
			      & \neg\varphi, X \:\:\:\:\:\:\:\: \psi, X
			\end{array}$

		\subsubsection*{Equivalence}
			$\begin{array}{rl}
				& \:\:\:\:\:\:\:\: \varphi \: \leftrightarrow \: \psi, X \\
			      \cline{2-2}
			      & \varphi, \psi, X \:\:\:\:\:\:\:\: \neg\varphi, \neg \psi, X
			\end{array}$
	\newline
	\newline
	\newline
	The final output of the Tableaux process is "False" when all branches are closed or a set of atomic formulae, when there exists a branch which is not closed.
	\newline

	For our usecases the functionality of the tableaux process shall be extended to achive better results, since if the branch is not closed, there are 
	additional calculations needed in order to verify that there is no contradiction, namely to verify that there is no contradiction on Term level.
	This verification can be done in different manners, depending on the algorithm type. The best way to think about it is to have the tableaux process
	return a list, where each element is the set of atomic formulae found in a specific branch. This way the atomic formulae for each branch are produced, and 
	afterwards can be used in different algorithms. This is just an example, a way of thinking about the problem the real implementation is much more space 
	efficient.

	\subsection{Tableaux implementation}
		The programming implementation of the tableaux method follows the standard tableaux process explaned above.

		\subsubsection{Definition: Marked Formula} 
			Let $\varphi$ be a formula and X be the set of accumulated formulae, then $\varphi $ is said to be marked as:
			\begin{itemize}
				\item true if and only if $\mathbb{T}\varphi  \in X $
				\item false if and only if $\mathbb{F}\varphi  \in X $
			\end{itemize}	

		First interesting design decision is to keep all true formulae in one data set, and all false formulae in another data set.
		This enables fast searches wheter a formula has been marked as true or false.
		There exist 8 important collections of data used for storing formulae.
		This means that each collection represents a set of formulae:
		\begin{itemize}
			\item formulas\_T\_ - contains only formulae marked as true
			\item formulas\_F\_ - contains only formulae marked as false, \\
				For example, if $\neg\varphi$ is encountered as an output of the tableaux step, then only $\varphi$ is inserted into the formula\_F\_
			\item contacts\_T\_ - contains only contacts formulae marked as true
			\item contacts\_F\_ - contains only contacts formulae marked as false
			\item zero\_terms\_T\_ - contains only formulae of type $\varphi \le \psi$ marked as true
			\item zero\_terms\_F\_ - contains only formulae of type $\varphi \le \psi$ marked as false
			\item measured\_less\_eq\_T\_ - contains only formulae of type $\varphi \le_m \psi$ marked as true
			\item measured\_less\_eq\_F\_ - contains only formulae of type $\varphi \le_m \psi$ marked as false
		\end{itemize}
	\newpage
\end{document}










































