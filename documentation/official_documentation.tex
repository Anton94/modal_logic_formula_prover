\documentclass{article}
\usepackage{amssymb}

\title{Satisfiability Of Modal Logic Formulas}
\date{2019-9-28}
\author{Martin Stoev, Anton Dudov}

\begin{document}
	\maketitle
	\pagenumbering{gobble}
	\newpage

	\section{Formula Representation}
	Let $\mathbb{V}ar$ be the set of variables:
	\[ \mathbb{V}ar = \{p_0, p_1, p_2 ... \}\]
	Let $\mathbb{C}_t$ be the set of Term constants:
	\[ \mathbb{C}_t = \{0, 1 \}\]
	Term recursive definition
	\begin{itemize}
		\item $a \in \mathbb{C}_t$ is a term
		\item $p \in \mathbb{V}ar$ is a term
		\item If x is a term, then $\bar{x}$ is a term as well
		\item If x and y are terms, then $x \sigma y $ is a term as well,\\
			where $\sigma \: \in \: \{\sqcap, \sqcup\}$
	\end{itemize}
	Let $\mathbb{C}_f$ be the set of formula constants:
	\[ \mathbb{C}_f = \{T, F \}\]
	Formula recursive definition
	\begin{itemize}
		\item $a \in \mathbb{C}_f$ is a formula
		\item If x and y are terms, then C(x, y) is a formula
		\item If x and y are terms, then $x \le y$ is a formula
		\item If x and y are terms, then $x \le_m y$ is a formula
		\item If $\phi$ is a formula, then $\neg \phi$ is a formula as well
		\item If $\phi$ and $\psi$ are formulas, then $\phi \: \sigma \: \psi $ is a formula as well,\\
			where $\sigma \in \{\vee, \wedge, \rightarrow, \leftrightarrow\}$
	\end{itemize}
	

	\section{Tableaux Process}
	The Tableaux process is decision procedure, which recursively breaks down a given formula into basic components 
	based on which a decision can be concluded. The recursive step which breaks down a formula creates one or two 
	new formulas, which in terms of their structure are simpler then the initial formula. Since the recursive step can create
	at most two new formulas, this means the recursive step will create at most two branches or a binary tree, where the nodes
	are the formulas and the links represent the recursive step.

	Contradiction may arise when in the same branch, on some step there exists a formula and the negation of the same formula.
	If in some branch there exists a contradiction, then that branch closes. If all branches close then the proof is complete.
	\newpage
\end{document}
